%% Latex markup and citations may be used here
% (abstract in German, at most 200 words. Example: )%\cite{loremIpsum})

% Lorem ipsum dolor sit amet, consectetur adipisici elit, sed eiusmod tempor
% incidunt ut labore et dolore magna aliqua. Ut enim ad minim veniam, quis
% nostrud exercitation ullamco laboris nisi ut aliquid ex ea commodi consequat.
% Quis aute iure reprehenderit in voluptate velit esse cillum dolore eu fugiat
% nulla pariatur. Excepteur sint obcaecat cupiditat non proident, sunt in culpa
% qui officia deserunt mollit anim id est laborum.

% Duis autem vel eum iriure dolor in hendrerit in vulputate velit esse molestie
% consequat, vel illum dolore eu feugiat nulla facilisis at vero eros et
% accumsan et iusto odio dignissim qui blandit praesent luptatum zzril delenit
% augue duis dolore te feugait nulla facilisi. Lorem ipsum dolor sit amet,
% consectetuer adipiscing elit, sed diam nonummy nibh euismod tincidunt ut
% laoreet dolore magna aliquam erat volutpat.

% Ut wisi enim ad minim veniam, quis nostrud exerci tation ullamcorper suscipit
% lobortis nisl ut aliquip ex ea commodo consequat. Duis autem vel eum iriure
% dolor in hendrerit in vulputate velit esse molestie consequat, vel illum dolore
% eu feugiat nulla facilisis at vero eros et accumsan et iusto odio dignissim qui
% blandit praesent luptatum zzril delenit augue duis dolore te feugait nulla
% facilisi.
F\"ur ein grundlegendes Verst\"andnis embryonaler Entwicklung ben\"otigen Entwicklungbiologen Wissen
\"ueber das Verhalten aller embryonaler Zellen. In diesem Zusammenhang ist tracking-by-assignment
ein vielgenutzter Ansatz für Zelltracking in der Embryogenese. Neue Methoden nutzen probabilitische
graphische Modelle um fehlerhafte Detektionen in der Segmentierung (Übersegmentierung) zu
beheben. Das Problem der Untersegmentierung hingegen bleibt ungelöst. Darum stellen wir zwei
Methoden vor, die sich des Problems der Untersegmentierung annehmen.

Zuerst nutzen wir Gaußsche Mischverteilungen für die Rekonstruktion von Zellidentitäten im Kontext
des Conservation Tracking. Dadurch können Zelltracks auch für Daten erstellt werden, die anfällig
für Untersegmentierung sind. Zudem präsentieren wir eine umfassende empirische Auswertung auf drei
anspruchsvollen $2d+t$ bzw. $3d+t$ Datensätzen.

Unser zweiter Beitrag, der einen neuen Ansatz im tracking-by-assignment Kontext darstellt, vereint
die Optimierung von Segmentierung und Zelltracking. Aus einer Auswahl von Segmentierungshypothesen,
die gegenseitig in Konkurrenz stehen, wählt unser Model diejenigen aus, die das beste Tracking
ergeben. Hierbei werden Zwangsbedingungen für das Tracking und eine konsistente Segmentierung in
einem Faktorgraphen formuliert, der global optimal gelöst wird. Erste Experimente liefern
verheißungsvolle Ergebnisse, die eine Fortführung des Projekts nahelegen.

%%% Local Variables: 
%%% mode: latex
%%% TeX-master: "../main"
%%% End: 
