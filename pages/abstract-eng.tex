%% Latex markup and citations may be used here
% (abstract in english, at most 200 words. Example: )% \cite{loremIpsum})

% Lorem ipsum dolor sit amet, consectetur adipisici elit, sed eiusmod tempor
% incidunt ut labore et dolore magna aliqua. Ut enim ad minim veniam, quis
% nostrud exercitation ullamco laboris nisi ut aliquid ex ea commodi consequat.
% Quis aute iure reprehenderit in voluptate velit esse cillum dolore eu fugiat
% nulla pariatur. Excepteur sint obcaecat cupiditat non proident, sunt in culpa
% qui officia deserunt mollit anim id est laborum.

% Duis autem vel eum iriure dolor in hendrerit in vulputate velit esse molestie
% consequat, vel illum dolore eu feugiat nulla facilisis at vero eros et
% accumsan et iusto odio dignissim qui blandit praesent luptatum zzril delenit
% augue duis dolore te feugait nulla facilisi. Lorem ipsum dolor sit amet,
% consectetuer adipiscing elit, sed diam nonummy nibh euismod tincidunt ut
% laoreet dolore magna aliquam erat volutpat.

% Ut wisi enim ad minim veniam, quis nostrud exerci tation ullamcorper suscipit
% lobortis nisl ut aliquip ex ea commodo consequat. Duis autem vel eum iriure
% dolor in hendrerit in vulputate velit esse molestie consequat, vel illum dolore
% eu feugiat nulla facilisis at vero eros et accumsan et iusto odio dignissim qui
% blandit praesent luptatum zzril delenit augue duis dolore te feugait nulla
% facilisi.
For a fundamental understanding of the development of embryos, biologists need knowledge of the fate
of each embryonic cell. For this purpose, tracking-by-assignment is a common approach for the
generation of cell tracks in embryogenesis. While recent work introduced a flexible graphical model
approach that can handle oversegmentation errors, the problem of undersegmentation, however, remains
unsolved to date. Therefore, we propose two methods that both attend to undersegmentation in the
context of tracking-by-assignment.

Firstly, we introduce Gaussian mixture models for cell identity reconstruction as part of the
conservation tracking method, which enables cell tracking in data prone to undersegmentation. In
addition we present an empirical evaluation on three challenging $2d+t$ and $3d+t$ embryogenesis
data sets, enriched by visualization and illustrative examples.

Our second, and more novel contribution is the introduction of a new tracking-by-assignment approach
that jointly optimizes both the segmentation and the tracking by choosing the best out of competing
segmentation hypotheses. Tracking and segmentation consistency constraints are formulated in terms
of a factor graph that is solved to global optimality. First experiments show promising results
of the model and suggest a continuation of the work.

%%% Local Variables: 
%%% mode: latex
%%% TeX-master: "../main"
%%% End: 
