\chapter{Integer Linear Programming}
\label{cha:app-ilp}
\settocdepth{chapter}

% \section{Integer Linear Programming}
\begin{mydef}[{\citealp[7]{vanderbei_07_linear}}]
    A \emph{linear program}~(LP) is an optimization problem over real valued \emph{decision variables}
    $x=\{x_i \in \mathbb{R}\}_{i=1,\hdots,N}$
    \begin{align}
        \label{eq:app-lp}
        &\max \; f(x) \\\nonumber
        &\;\;\;\;\;\; \st \\ \nonumber
        &g_1(x) \le b_1 \\ \nonumber
        &\;\;\;\;\;\; \vdots \\\nonumber
        &g_K(x) \le b_K
    \end{align}
    with the objective
    \begin{align}
        f(x) = \sum_i^Nc_ix_i
    \end{align}
    and the constraints
    \begin{align}
        g_j(x) = \sum_i^Na_{ji}x_i \; \forall \; 1 \le j \le K
    \end{align}
    linear in $x$.
\end{mydef}

With matrix formulation, \cref{eq:app-lp} becomes
\begin{align}
    &\max \; c^{\intercal}x \\\nonumber
    &\;\;\;\;\;\; \st \\ \nonumber
    &  Ax \le b.
\end{align}
Then, an \emph{integer linear program}~(ILP) is a linear program in which a subset or all of the
decision variables are restricted to integer numbers $\mathbb{Z}$ or subsets
thereof~\citep[385]{vanderbei_07_linear}. Note that, by this definition, a binary assignment
problem, $x_i\in\{0,1\} \subset \mathbb{Z}$, is an integer linear program. For further readings, we
refer the reader to \citet{vanderbei_07_linear, padberg_10_linear} for linear programming in general
and to \citet[Chapter~5]{cook_12_pursuit} and \citet[Chapter~23]{vanderbei_07_linear} for integer
linear programming in particular.

\chapter{Random Forests}
\label{cha:app-rf}
A random forest classifier is an ensemble of $n$ randomized decision trees. Randomness is injected
during tree construction from labelled training data by using
\begin{enumerate}
      \item a random subset of the training data (bags~\citealp{breiman_96_bagging}) for each tree
    and
      \item a random subset of features at each decision node of a tree.
\end{enumerate}
Typically, the \emph{Gini impurity measure}~\citep{breiman_84_classification,ceriani_12_origins} is
used as a splitting criterion.

For prediction, a sample $s$ is passed down each of the trees and each tree votes for a label. The
fraction of trees which predicted label $l$ can be interpreted as the probability of the label of $s$
being $l$. For further readings, we refer the reader to \citet{breiman_01_random,biau_12_analysis}.

\section{Pixel Classification and Segmentation with Random Forests}
The \emph{Interactive Learning and Segmentation Toolkit}~\citep{sommer_11_ilastik} utilizes the
power of random forests for pixelwise prediction. First, the user selects training samples by pencil
strokes in a view on the data. Secondly, features are calculated on a selection of features, \eg
Gausian gradient magnitude, defined by the user. Then, a random forest classifier is trained on the
annotated training samples and selected features. This classifier then can be used for prediction on
the whole data set. The output is a \emph{probability map} that contains the probability for each
label at each pixel position. Finally, a segmentation is obtained by thresholding the probability
for the desired label or by taking the $\argmax$ of the labels at each pixel position.


\chapter{Classifier Feature Specification}
\label{cha:app-joint-classifier-features}
Classification results depend on the selection of an appropriate set of features. Here, the
feature selection for the classifiers in \cref{sec:joint-classifiers} is summarized.

\section{Count Classifier}
\label{sec:app-joint-classifier-count}
\cref{tab:joint-classifier-count} shows the range of vigra~\citep{koethe_08_vigra} region
features used for the count classifier.
\begin{table}[h]
    \centering
    \begin{tabular}{c}
        \toprule
        \begin{minipage}{\textwidth}
            \begin{multicols}{3}
                \begin{itemize}
                    \NoBulletItem \mbox{Central<PowerSum<2>\ >}
                    \NoBulletItem \mbox{Central<Powersum<3>\ >}
                    \NoBulletItem \mbox{Central<PowerSum<4>\ >}
                    \NoBulletItem \mbox{Weighted<PowerSum<0>\ >}
                    \NoBulletItem Count
                    \NoBulletItem Histogram
                    \NoBulletItem Kurtosis
                    \NoBulletItem Maximum
                    \NoBulletItem Mean
                    \NoBulletItem Minimum
                    \NoBulletItem Quantiles
                    \NoBulletItem RegionAxes
                    \NoBulletItem RegionRadii
                    \NoBulletItem Skewness
                    \NoBulletItem Sum
                    \NoBulletItem Variance
                \end{itemize}
            \end{multicols}
        \end{minipage} \\ \\
        \bottomrule
    \end{tabular}
    \caption[Feature Selection: Count Classifer]{Feature selection for the joint segmentation and
        tracking count classifier.}
    \label{tab:joint-classifier-count}
\end{table}

\section{Region Classifier}
\label{sec:app-joint-classifier-region}
\cref{tab:joint-classifier-region} shows the range of vigra region features used for the region
classifier.
\begin{table}[h]
    \centering
    \begin{tabular}{c}
        \toprule
        \begin{minipage}{\textwidth}
            \begin{multicols}{3}
                \begin{itemize}
                    \NoBulletItem{RegionRadii}
                    \NoBulletItem{Mean}
                    \NoBulletItem{Count}
                    \NoBulletItem{Variance}
                    \NoBulletItem{Sum}
                \end{itemize}
            \end{multicols}
        \end{minipage} \\ \\
        \bottomrule
        \end{tabular}
    \caption[Feature Selection: Region Classifer]{Feature selection for the joint segmentation and
        tracking region classifier.}
    \label{tab:joint-classifier-region}
\end{table}

\section{Division Classifier}
\label{sec:app-joint-classifier-division}
The division classifier is applied to modified vigra region features. The modifiers and region
features are listed in \cref{tab:joint-classifier-division-modifiers} and
\cref{tab:joint-classifier-division-features} respectively.

\begin{table}[h]
    \renewcommand*{\arraystretch}{2.5}
    \centering
    \begin{tabular}{ll}
        \toprule
        Modifier $m$ & Effect $m(p, f^{(1)},f^{(2)}); \;\; p, f^{(1)}, f^{(2)} \; \in \; \mathbb{R}^N$ \\ 
        \midrule
        MaxParentRatio & $\max_{i\in\{1,2\}}\text{Ratio}(p,f^{(i)})$ \\
        MaxParentSquaredDifference & $\max_{i\in\{1,2\}}\text{SquaredDiff}(p, f^{(i)})$ \\
        MeanParentRatio & $\frac{1}{2}\sum_{i=1}^2\text{Ratio}(p, f^{(i)})$ \\
        MeanParentSquaredDifference & $\frac{1}{2}\sum_{i=1}^2\text{SquaredDifference}(p, f^{(i)})$ \\
        MinParentRatio & $\min_{i\in\{1,2\}}\text{Ratio}(p,f^{(i)})$ \\
        MinParentsquaredDifference &$\min_{i\in\{1,2\}}\text{SquaredDiff}(p, f^{(i)})$ \\
        RatioParentSquaredDifference & $\frac{\min_{i\in\{1,2\}}\text{SquaredDiff}(p, f^{(i)})}{\max_{i\in\{1,2\}}\text{SquaredDiff}(p, f^{(i)})}$ \\
        \bottomrule
    \end{tabular}
    \caption[Modifiers for vigra region features]{Division modifiers for vigra region features. $p$
        is a feature of the parent cell, $f^{(1)}$ and $f^{(2)}$ are features of the children cells.}
    \label{tab:joint-classifier-division-modifiers}
\end{table}

\newlength\tablemodifierheight
\settototalheight\tablemodifierheight{\parbox{\linewidth}{RegionCenter}}
\begin{table}[h]
    \centering
    \renewcommand*{\arraystretch}{1.5}
    \begin{tabular}{lc}
        \toprule
        Modifier & Features \\ \midrule
        MaxParentRatio &
        \raisebox{\tablemodifierheight-\height}{
            \begin{minipage}{0.5\textwidth}
                \begin{multicols}{2}
                    \begin{itemize}
                        \NoBulletItem Mean
                        \NoBulletItem Count
                        \NoBulletItem Variance
                        \NoBulletItem Sum
                    \end{itemize}
                \end{multicols}
            \end{minipage}
        } \\ & \\
        MaxParentSquaredDifference &
        \raisebox{\tablemodifierheight-\height}{
            \begin{minipage}{0.5\textwidth}
                \begin{multicols}{2}
                    \begin{itemize}
                        \NoBulletItem Mean
                        \NoBulletItem Count
                        \NoBulletItem Variance
                        \NoBulletItem Sum
                    \end{itemize}
                \end{multicols}
            \end{minipage}
        } \\ & \\
        MeanParentRatio &
        \raisebox{\tablemodifierheight-\height}{
            \begin{minipage}{0.5\textwidth}
                \begin{multicols}{2}
                    \begin{itemize}
                        \NoBulletItem Mean
                        \NoBulletItem Count
                        \NoBulletItem Variance
                        \NoBulletItem Sum
                    \end{itemize}
                \end{multicols}
            \end{minipage}
        } \\ & \\
        MeanParentSquaredDifference &
        \raisebox{\tablemodifierheight-\height}{
            \begin{minipage}{0.5\textwidth}
                \begin{multicols}{2}
                    \begin{itemize}
                        \NoBulletItem Mean
                        \NoBulletItem Count
                        \NoBulletItem Variance
                        \NoBulletItem Sum
                    \end{itemize}
                \end{multicols}
            \end{minipage}
        } \\ & \\
        MinParentRatio &
        \raisebox{\tablemodifierheight-\height}{
            \begin{minipage}{0.5\textwidth}
                \begin{multicols}{2}
                    \begin{itemize}
                        \NoBulletItem Mean
                        \NoBulletItem Count
                        \NoBulletItem Variance
                        \NoBulletItem Sum
                    \end{itemize}
                \end{multicols}
            \end{minipage}
        } \\ & \\
        MinParentSquaredDifference &
        \raisebox{\tablemodifierheight-\height}{
            \begin{minipage}{0.5\textwidth}
                \begin{multicols}{2}
                    \begin{itemize}
                        \NoBulletItem Mean
                        \NoBulletItem Count
                        \NoBulletItem Variance
                        \NoBulletItem Sum
                    \end{itemize}
                \end{multicols}
            \end{minipage}
        } \\ & \\
        RatioParentSquaredDifference &
        \raisebox{\tablemodifierheight-\height}{
            \begin{minipage}{0.5\textwidth}
                \begin{multicols}{2}
                    \begin{itemize}
                        \NoBulletItem Regioncenter
                        \NoBulletItem Mean
                        \NoBulletItem Count
                        \NoBulletItem Variance
                        \NoBulletItem Sum
                    \end{itemize}
                \end{multicols}
            \end{minipage}
        } \\ & \\
        \bottomrule
    \end{tabular}
    \caption[Feature Selection: Move Classifier]{Feature selection for the joint segmentation and
        tracking division classifier. Each modifier is applied to the features listed in the associated field.}
    \label{tab:joint-classifier-division-features}
\end{table}

\section{Move Classifier}
\label{sec:app-joint-classifier-move}
The move classifier is applied to modified vigra region features. The modifiers and region features
are listed in \cref{tab:joint-classifier-move-modifiers,tab:joint-classifier-move-features}
respectively.
\begin{table}[h]
    \renewcommand*{\arraystretch}{2.5}
    \centering
    \begin{tabular}{ll}
        \toprule
        Modifier $m$ & Effect $m(f^{(1)},f^{(2)}); \;\; f^{(1)}, f^{(2)} \; \in \; \mathbb{R}^N$ \\ 
        \midrule
        AbsDiff & $m(f^{(1)}, f^{(2)}) = \sum_{i=1}^N|f^{(1)}_i - f^{(2)}_i|$ \\
        Ratio & $N=1 : m(f^{(1)}, f^{(2)}) = min(\frac{f^{(1)}}{f^{(2)}},\frac{f^{(2)}}{f^{(1)}})$ \\
        SquaredDiff & $m(f^{(1)}, f^{(2)}) = \sum_{i=1}^N(f^{(1)}_i - f^{(2)}_i)^2$ \\
        SqrtSquaredDiff & $m(f^{(1)}, f^{(2)}) = \sqrt{\sum_{i=1}^N(f^{(1)}_i - f^{(2)}_i)^2}$ \\
        \bottomrule
    \end{tabular}
    \caption[Modifiers for vigra region features]{Move modifiers for vigra region
        features. $f^{(1)}$ and $f^{(2)}$ are features of cells involved in a move.}
    \label{tab:joint-classifier-move-modifiers}
\end{table}

\begin{table}[h!]
    \centering
    \renewcommand*{\arraystretch}{1.5}
    \begin{tabular}{lc}
        \toprule
        Modifier & Features \\ \midrule
        AbsDiff &
        \raisebox{\tablemodifierheight-\height}{
            \begin{minipage}{0.5\textwidth}
                \begin{multicols}{2}
                    \begin{itemize}
                        \NoBulletItem RegionCenter
                        \NoBulletItem Count
                        \NoBulletItem Mean
                        \NoBulletItem Variance
                        \NoBulletItem Sum
                    \end{itemize}
                \end{multicols}
            \end{minipage}
        } \\ & \\
        Ratio &
        \raisebox{\tablemodifierheight-\height}{
            \begin{minipage}{0.5\textwidth}
                \begin{multicols}{2}
                    \begin{itemize}
                        \NoBulletItem Count
                        \NoBulletItem Mean
                        \NoBulletItem Sum
                    \end{itemize}
                \end{multicols}
            \end{minipage}
        } \\ & \\
        SqartSquaredDiff &
        \raisebox{\tablemodifierheight-\height}{
            \begin{minipage}{0.5\textwidth}
                \begin{multicols}{2}
                    \begin{itemize}
                        \NoBulletItem RegionCenter
                        \NoBulletItem Count
                        \NoBulletItem Mean
                        \NoBulletItem Variance
                        \NoBulletItem Sum
                    \end{itemize}
                \end{multicols}
            \end{minipage}
        } \\ & \\
        SquaredDiff &
        \raisebox{\tablemodifierheight-\height}{
            \begin{minipage}{0.5\textwidth}
                \begin{multicols}{2}
                    \begin{itemize}
                        \NoBulletItem RegionCenter
                        \NoBulletItem Count
                        \NoBulletItem Mean
                        \NoBulletItem Variance
                        \NoBulletItem Sum
                    \end{itemize}
                \end{multicols}
            \end{minipage}
        } \\ & \\
        \bottomrule
    \end{tabular}
    \caption[Feature Selection: Move Classifier]{Feature selection for the joint segmentation and
        tracking move classifier. Each modifier is applied to the features listed in the associated field.}
    \label{tab:joint-classifier-move-features}
\end{table}

        



%%% Local Variables: 
%%% mode: latex
%%% TeX-master: "../main"
%%% End: 
