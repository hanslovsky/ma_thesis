\section{Pixel Classification}
\label{sec:pixel-classification}

Pixel classification is the task of assigning each pixel $p$ in the image $I$ to one of $n$ possible foreground
classes or background. A classifier
\begin{align}
    \mathcal{C}\colon \mathcal{R}^d &\to [0,1]^{n+1} \\
    x^p &\mapsto y^{n+1}    
\end{align}
maps from the d-dimensional feature space to the interval $[0,1]^{n+1}$ with $y_k^p$ representing the
probability of pixel $p$ belonging to the $k$th class. Hence,
\begin{align}
    \sum_{i=1}^{n+1}y_i^p = 1
\end{align}
must hold. The class label for the pixel is the label with highest probability given by
\begin{align}
    l^p = \argmax y^p, 
\end{align}
also known as ``winner takes it all''. Alternatively, in a two-class classification problem, \ie one
foreground \vs background, using a threshold $t \in (0,1)$ as decision boundary also yields a
segmentation, that -- in case of $t=0.5$ -- is equivalent to the $\argmax$ decision.

In general, any classifier that maps to $[0,1]$ can be used for pixel classification. In practice,
the \emph{random forest} classifier is a state-of-the-art classification algorithm, that is part of
the ``interactive learning and segmentation toolkit''
(ILASTIK)\footnote{http://www.ilastik.org}. ILASTIK calculates pixel features using image filters,
such as Gaussian Smoothing, Guassian Gradient Magnitude or Structure Tensor.
For training the random forest classifier, the user needs to give examples for each of the possible
classes. ILASTIK then trains the classifier based on these examples.

%%% Local Variables: 
%%% mode: latex
%%% TeX-master: "../../main"
%%% End: 
