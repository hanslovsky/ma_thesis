\chapter{Conclusion}
\label{cha:conclusion}
We presented two cell tracking methods which both address the problem of undersegmentation in cell
tracking-by-assignment. The first of these methods, namely Gaussian mixture models for the
conservation tracking method~(\cref{cha:GMM}), effectively extends the conservation tracking factor
graph and uses the information about the number of cells per connected component to successfully
reconstruct cell identities by the means of Gaussian mixture models. A cell tracking graphical model
is formulated on the subset of reconstructed cells that enables the conservation tracking to track
cells in dense populations and to effectively cope with undersegmentation errors. Furthermore, the
tracking result for segmentations without mergers is comparable to the state of the art, chain graph
tracking~\citep{kausler_12_discrete}, which does not have the ability to detect mergers or
reconstruct the identities of the individual cells. Thus, the conservation tracking -- with both a
similar performance in tracking without mergers and identity reconstruction -- has the edge over
chain graph tracking.

Our second proposed method, joint segmentation and tracking~(\cref{cha:joint}), goes beyond existing
tracking-by-assignment methods and unifies segmentation and tracking in allowing the tracking to
modify the segmentation and choose of the offered, potentially conflicting segmentation hypotheses
those ones that best fit the global tracking. In the fashion of chain graph
tracking~\citep{kausler_12_discrete} and conservation tracking~\citep{schiegg_13_conservation}, we
present a factor graph representation of the problem, which implements both tracking and
segmentation consistency constraints. While, as yet, the proposed method is not competitive, first
results look promising and, with the suggested modifications in \cref{subsec:joint-continuation}, we
are confident that the joint segmentation and tracking model will eventually produce state of the
art results and moreover be capable of tracking cells in conditions, where other methods fail due to
undersegmentation errors, namely very dense populations and low quality data.

Future work focuses on remodeling the joint segmentation and tracking method in order to produce
comparable results. In addition, new evaluation measures need to be introduced as -- in contrast to
chain graph and conservation tracking, which both are applied to fixed segmentations -- the final
segmentation result needs to be evaluated and compared with other methods as well.

%%% Local Variables: 
%%% mode: latex
%%% TeX-master: "../main"
%%% End: 
