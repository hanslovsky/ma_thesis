\section{Graphical Model}
\label{sec:joint-graphical-model}
\begin{figure}
\begin{center}
    \begin{tikzpicture}[auto, thick, on grid,scale=1.8,every node/.style={scale=1.5, thick}]
        % Physical layer nodes

\newcommand{\distancebetween}{80}

\begin{scope}[
    yshift=\distancebetween,
    every node/.append style={yslant=0.5,xslant=-1},yslant=0.5,xslant=-1
    ]
    \begin{pgfonlayer}{upper}
        % \draw[black,thin, fill=black!70!blue, opacity=1, fill opacity=0.2] (-3,1.0) rectangle (0.1,3.2);
        \foreach \place/\x/\id/\t in {{(-2.7,2.9)/1/1/1}, {(-2.7,2.1)/2/2/1},{(-0.2,2.1)/3/5/1},
            {(-2.7,1.3)/4/3/1}, {(-1.25,1.7)/5/4/1}} {
            \node[hypotheses_three_objects, draw=blue!100, fill=blue!10, label=center:\footnotesize{$\displaystyle X_{\id}^{\t}$},
            ultra thick] (a\x) at \place {\phantom{\small{0}}};
        }

        % \node[unaryfac, above of=a3,yshift=-3mm,label=above right:$\psi_{\text{det}}$] (u1) {};
        % \path[unaryfac] (a3) edge (u1);
        % \node[unaryfac, below right of=a5,xshift=-2.2mm,yshift=2.2mm)] (u2) {};
        % \path[unaryfac] (a5) edge (u2);
        % \node[unaryfac, above left of=a1,yshift=-3mm] (u3) {};
        % \path[unaryfac] (a1) edge (u3);
        % \node[unaryfac, above left of=a2,yshift=-3mm] (u4) {};
        % \path[unaryfac] (a2) edge (u4);
        % \node[unaryfac, above left of=a4,yshift=-3mm] (u5) {};
        % \path[unaryfac] (a4) edge (u5);

        % \coordinate[draw=black!0, left of=a4] (helper1) {};
        % \coordinate[draw=black!0, left of=a3] (helper2) {};
        % \coordinate[draw=black!0, left of=helper2] (helper3) {};
        \node[conflict, left of=a4] (helper1) {};
        \node[conflict, left of=a3, xshift=1mm] (helper2) {};
        \node[conflict, yshift=5, label=above right:$\psi_{\text{det}}$] (helper4) at ($(a1)!0.5!(a3)$) {};
        \node[count, left of=helper2] (helper3) {};
        
        % Physical layer links
        \path[conflict] (a1) edge[bend left=0] (helper4);
        \path[conflict] (helper4) edge[bend left=0] (a3);
        \path[conflict] (a5) edge (helper2);
        \path[conflict, bend left=10] (a4) edge (helper2);
        
        \path[conflict] (a5) edge[bend left=50] (helper1);
        \path[conflict] (helper1) edge[bend left=50] (a2.west);

        \path[conflict] (a3) edge[bend left=60] coordinate[pos=0.65](helpercoord1) (helper1);
        % \gettikzxy{(helpercoord1)}{\ax}{\ay}; \coordinate (newfit) at (\ax,\ay);
        % \path[thick,black] (helpercoord1) edge (helper1);
        \path[conflict] (a3) edge (helper2);

        \path[count] (a1) edge (helper3);
        \path[count] (a2) edge (helper3);
        \path[count] (a3) edge[bend right=20] (helper3);
        \path[count] (a4) edge (helper3);
        \path[count] (a5) edge (helper3);
    \end{pgfonlayer}

    \begin{pgfonlayer}{bgupper}
        \node[rectangle, color=black,thick, fill=hypothesesbackground!30, opacity=0.8, draw=black,
        fit=(a1) (a2) (a3) (a4.south) (a5) (helper1) (helpercoord1), inner sep=5mm, opacity=0.8, label={[xshift=7mm]{$t=1$}}]
        (bgup) {};
    \end{pgfonlayer}

    % \path[conflict] (helper1) edge[bend left=80] (a2.north west);

    % \path[conflict] (a2) edge
\end{scope}





\begin{scope}
    [
    every node/.append style={yslant=0.5,xslant=-1},yslant=0.5,xslant=-1
    ]
    \begin{pgfonlayer}{lower}
        % \draw[black,thin, fill=black!70!blue, opacity=1, fill opacity=0.2] (-3,1.0) rectangle (0.1,3.2);
        % Overlay layer nodes
        \foreach \place/\i/\id/\t in {{(-2.7,2.9)/1/6/2},{(-0.2,2.1)/2/8/2},{(-2.7,1.3)/3/7/2}}
        \node[hypotheses_three_objects, draw=blue!100, fill=blue!10, label=center:\footnotesize{$\displaystyle X_{\id}^{\t}$},
        ultra thick] (b\i) at \place {\phantom{\small{0}}};
        % \node[unaryfac, above of=b2,yshift=-3mm,label=above right:$\psi_{\text{det}}$] (ub1) {};
        % \path[unaryfac] (b2) edge (ub1);
        % \node[unaryfac, above left of=b1,yshift=-3mm] (ub2) {};
        % \path[unaryfac] (b1) edge (ub2);
        % \node[unaryfac, above left of=b3,yshift=-3mm] (ub3) {};
        % \path[unaryfac] (b3) edge (ub3);
        
        % Overlay layer links
        \path[conflict] (b1) edge (b2);
        \path[conflict] (b3) edge (b2);
        \node[conflict] (c1) at ($(b1)!0.5!(b2)$) {};
        \node[conflict] (c2) at ($(b2)!0.5!(b3)$) {};
        \node[count, label={[yshift=2mm]left:$\psi_{\text{count}}$}] (c3) at ($(b1)!0.5!(b3)!0.3!(b2)$) {};

        \path[count] (b1) edge (c3);
        \path[count] (b2) edge (c3);
        \path[count] (b3) edge (c3);
    \end{pgfonlayer}

    \begin{pgfonlayer}{bglower}
        % \draw
        % let \p1=(bgup), \p2=($(bgup.north) - (bgup.south)$), \p3=($(bgup.west) - (bgup.east)$)
        % in node[rectangle, fill=red!30,minimum height={veclen(\x2,\y2)}, minimum
        % width={veclen(\x3,\y3)}] at(\x1,\y1) (bgdown) {};
        \node[rectangle, black,thick, fill=hypothesesbackground!30, opacity=0.8, draw=black,
        fit=(a1) (a2) (a3) (a4.south) (a5) (helper1) (helpercoord1), inner sep=5mm, shift=($(b2) - (a3)$), label={[xshift=7mm]{$t=2$}}] (bgdown) {};
    \end{pgfonlayer}
    
\end{scope}

\begin{pgfonlayer}{transition}
    \node[hypotheses_one_object, fill=hypothesesoneobject!20, ultra thick, label=center:$Y_{5,8}^1$]
    (as1) at ($(a3)!0.6!(b2)$) {\phantom{$t$}}; \node[transfac,opacity=1.0,
    label=right:$\psi_{\text{out}}$] (out) at ($(a3)!0.5!(as1)$) {};
    \path[transfac,opacity=1.0] (a3) edge (out); \path[transfac,opacity=1.0] (out) edge (as1);


    \node[transfac,opacity=1.0, label=right:$\psi_{\text{in}}$] (in) at ($(as1)!0.5!(b2)$) {};
    \path[transfac,opacity=1.0] (as1) edge (in);
    \path[transfac,opacity=1.0] (in) edge (b2.center);

    % edges not drawn, only indicated
    \coordinate[above right=of in.south west] (ind1);
    \coordinate[above left=of in.south east] (ind2);
    \path[transfac,opacity=0.5,dashed] (in.north) edge (ind1);
    \path[transfac,opacity=0.5,dashed] (in.north) edge (ind2);

    \coordinate[below right=of out.north west] (ind3);
    \coordinate[below left=of out.north east] (ind4);
    \path[transfac,opacity=0.5,dashed] (out) edge (ind3);
    \path[transfac,opacity=0.5,dashed] (out) edge (ind4);
\end{pgfonlayer}


% \begin{scope}
%     \node[hypotheses_three_objects, draw=blue!100, fill=blue!20, label=center:\footnotesize{$\displaystyle X_{\id}^{\t}$},
%     ultra thick]
% \end{scope}

%%% Local Variables: 
%%% mode: latex
%%% TeX-master: "../../../main"
%%% End: 

    \end{tikzpicture}
\end{center}
\caption[Joint segmentation and tracking factor graph for two connected components.]{Joint
    segmentation and tracking factor graph for two connected components at times $t$ and $t+1$: For
    the purpose of readability, the connected component indices have been dropped from detection
    (blue circles) and transition (yellow circles) variables. Still, the example remains
    unambiguous. Detection factors and count factors are denoted by red and green squares
    respectively. At the same time, the red factors represent maximal cliques (or conflict sets) in
    the conflict graph of regions.  From the black incoming and outgoing factors as well as from the
    transition nodes, only a representative sample is shown. Note that each of the detection
    variables has both an incoming and an outgoing factor. Furthermore, the dashed lines at the
    incoming and outgoing factors symbolize additional transitions. This excerpt of the factor graph
    illustrates the complexity of the model.}
\label{fig:joint-factor_graph}
\end{figure}

Based on the oversegmentation described in \cref{sec:joint-oversegmentation},
% a meaningful tracking is generated. We 
we introduce a factor graph which models competing (intra-frame)
relations between potential cell segmentations which overlap in
space %from the same connected component
as well as possible inter-frame hypotheses between regions of adjacent time steps. Factors are
introduced to encourage solutions from local classifiers and, at the same time, guarantee
consistency due to
%The factors represent prior belief in the correctness of the tracking. 
%In addition, 
linear constraints. That is, impossible configurations are disallowed, \eg a
cell dividing into three daughters. % Building the graphical model corresponds to step
% (IV) in \cref{fig:joint-pipeline} where the result after running inference on the proposed graphical
% model is marked. 
%Inference will lead to the tracking result as shown in step (IV) (both referring to
%Fig.~\ref{fig:pipeline}). 
The construction of the factor graph, the meaning of contained factors, and random variables are
described in detail in this section.
% We will refer to the toy example depicted in
% \cref{fig:joint-pipeline} as a running example.

\subsection{Random Variables}
%The factor graph models two types of binary random variables: 
In order to build the factor graph for joint segmentation and tracking, we first introduce two types
of binary random variables, \emph{detection} variables and \emph{transition} variables.
%Firstly,
In particular, each possible cell segmentation (region) gets assigned a \emph{detection} variable
$X_{i\alpha}^t \in \{0,1\}$, where $i$ is the connected component the region is contained in,
$\alpha$ is the identifier of the region, and $t$ is the time step. The set of indices of all regions
contained in a connected component is denoted by $I_i^t$ and $\mathcal{C}_{i\alpha}^t$ is the set of
indices of all regions that are part of the conflict set identified by segment
$\alpha$. Furthermore, in accordance with \cref{def:joint-cardinality}, $|X_{i\alpha}^t|$ is the
cardinality of the region associated with $X_{i\alpha}^t$. Finally, $\mathcal{X}$ denotes the set of
all detection variables.

Secondly, variables $Y_{i\alpha,j\beta}^{t} \in \{0,1\}$ for each possible inter-frame transition
between two regions in adjacent time steps $t$ and $t+1$ are added. $\mathcal{Y}_{\rightarrow
    j\beta}^{t}$ and $\mathcal{Y}_{i\alpha \rightarrow}^{t}$ denote the set of all incoming
transitions for $X_{j\beta}^{t+1}$ and the set of all outgoing transitions for $X_{i\alpha}^{t}$
respectively. In addition, $\mathcal{Y}$ is the set of all assignment variables.
%tracking between two targets in adjacent timesteps. 
%These random variables can be inerpreted as
%intra-timestep and inter-timestep hypotheses, respectively. 
% In our illustrative example in \cref{fig:joint-pipeline}, %exemplary random variables are
% one detection variable is $X_{\{45\}\{4\}}^{t+1}$, referring to region $4$ in the connected
% component formed by regions $4$ and $5$ at time $t+1$.  $Y_{\{123\}\{23\},\{45\}\{4\}}^{t}$ is an
% exemplary inter-frame transition variable, where the indices mean that
% %An inter-frame , referring to the transition of
% region $23$ in connected component $123$ at time $t$ may be associated with region $4$ in connected
% component $45$ at time $t+1$.

% Inter-frame hypotheses are extracted by a thresholded nearest neighbor search for connected components between
% adjacent timesteps, both in forward and backward direction. Each region within a connected component
% at time $t$ is paired with all regions in the neighboring connected components at time $t+1$. These
% are the hypotheses that are represented by $Y_{i\alpha,j\beta}^{t,t+1}$  in the factor graph.

\subsection{Factors}
We continue the construction of our graphical model by adding factors.
Factors %in the factor graph relate the random variables to each other and
may disallow specific configurations (\cf \cref{subsec:joint-linear-constraints}) and score possible
configurations of their associated variables based on
%, allowing for choosing the best feasible configuration during
%inference. Each factor represents a potential whose value is determined by the states of the random
%variables belonging to that factor. In order to weigh factors against each other, there is a design
%parameter $w_{\text{class}}$ for each factor class. The 
probabilities $\hat{P}$ that are here determined by probabilistic discriminative classifiers using
local evidence $f_{i\alpha}^t$.  In the following, intra-frame factors (detection and count factor)
and inter-frame factors (outgoing and incoming factors) are described. For adjustment, these factors
are weighted against each other by the corresponding design parameter $w$.


% For each detection variable, a unary \emph{detection factor}
% \begin{align}
%     \label{eq:psi-det}
%     \phi_{\text{det}}(X_{i\alpha}^t) & = E_{\text{det}}(X_{i\alpha}^t) \\
%     E_{\text{det}}(X_{i\alpha}^t) & = -w_{\text{det}}\log\left(\hat{P}_\mathrm{det}\left(X_{i\alpha}^t = k|f_{i\alpha}^t\right)\right),\; k\in\{0,1\}
% \nonumber
% \end{align}
% gives a prior probability that the associated region is a true segmentation of a cell. 

For each conflict set $\mathcal{C}_{i\nu}^t$ with regions 
\begin{align}
    \label{eq:psi-det}
    \mathcal{K}_{i\nu}^t = \{X_{i\kappa}^t : \kappa\in\mathcal{C}_{i\nu}^t\},
\end{align}
and local evidence $f_{i\alpha}^t$ for all of these regions,
\begin{align}
    \xi_{i\nu}^t = \{f_{i\kappa}^t : \kappa\in\mathcal{C}_{i\nu}^t\},
\end{align}
a higher order \emph{detection factor}
\begin{align}
    \psi_{\text{det}}(\mathcal{K}_{i\nu}^t, \xi_{i\nu}^t) =
    e^{-E_{\text{det}}(\mathcal{K}_{i\nu}^t, \xi_{i\nu}^t)},
\end{align}
\begin{subnumcases}{E_{\text{det}}(\mathcal{K}_{i\nu}^t, \xi_{i\nu}^t)=}
    A(\mathcal{K}_{i\nu}^t, \xi_{i\nu}^t), &\text{if} $\sum_{X \in \mathcal{K}_{i\nu}^t} X = 1$ \label{eq:joint-factor-one}\\ 
    B(\mathcal{K}_{i\nu}^t, \xi_{i\nu}^t), &\text{if} $\sum_{X \in \mathcal{K}_{i\nu}^t} X = 0$ \label{eq:joint-factor-zero}\\ 
    \infty &\text{otherwise} \label{eq:joint-factor-segmentation-constraint}
\end{subnumcases}
with
\begin{align}
    A(\mathcal{K}_{i\nu}^t, \xi_{i\nu}^t) &=-w_{\text{det}} \sum_{X_{i\alpha}^t \in \mathcal{K}_{i\nu}^t}
    \mathds{1} \left(X_{i\alpha}^t =
        1\right)\log\left(\hat{P}(X_{i\alpha}^t=1|f_{i\alpha}^t)\right), \label{eq:joint-factor-one-A}
    \\
    B(\mathcal{K}_{i\nu}^t, \xi_{i\nu}^t) &= -w_{\text{det}} \max_{X_{i\alpha}^t\in\mathcal{K}_{i\nu}^t}
    \log\left(\hat{P}(X_{i\alpha}^t = 0|f_{i\alpha}^t)\right) + c_{\text{opp}} \label{eq:joint-factor-zero-B}
\end{align}
assigns high energies to configurations of a conflict set $\mathcal{C}_{i\nu}^t$ which contain an
active, unlikely segmentation candidate, as determined by a probabilistic discriminative
classifier~(\cf \cref{sec:joint-classifiers}, $\hat{P}_\mathrm{det}$%\left(X_{i\alpha}^t =
% k|f_{i\alpha}^t\right)$
, with $f_{i\alpha}^t$ denoting local evidence, and low energies to
configurations that follow the prior belief of the classifier. In this connection, the design
parameter $c_{\text opp}$, \emph{opportunity cost}, is a bias towards active cells, \ie in case of
low probability for all regions in the conflict set, a high opportunity cost pushes the graphical
model towards activating one of the regions, nonetheless.  Furthermore, within each conflict set, a
penalty is paid only once. This is due to the \emph{segmentation conflict}
constraint~(Equation~\ref{eq:joint-factor-segmentation-constraint}, or $\mathfrak{C}_1$ in
\cref{tab:constraints}), which allows only one active cell per conflict set. This is ensured by
selecting the active region with the indicator function
\begin{align}
    \mathds{1}(\text{statement}) =
    \begin{cases}
        1, & \text{statement is true} \\
        0, & \text{statement is false}
    \end{cases} 
\end{align}
in Equation~\ref{eq:joint-factor-one-A} and by taking the maximum in
Equation~\ref{eq:joint-factor-zero-B}. In \cref{fig:joint-pipeline}, the potential $E_{\text det}$
ideally obtains a high value for the green and yellow regions in the initial oversegmentation of the
upper connected component and ideally gets a low score for the merged yellow region, since it
better represents an entire cell.

Moreover, to exploit further local evidence, a higher-order \emph{count factor}
\begin{align}
    \label{eq:psi-count}
    \psi_{\text{count}}(\{X_{i\alpha}^t\}_{\alpha \in
        I_i^t},f_{i}^t)&=e^{-E_{\text{count}}(\{(X_{i\alpha}^t,f_{i\alpha}^t)\}_{\alpha \in I_i^t})}, \\
    E_{\text{count}}(\{X_{i\alpha}^t\}_{\alpha \in I_i^t},f_{i}^t) &=
    -w_{\text{count}}\log\left(\hat{P}_{\text{count}}\left(\sum_{X \in \{X_{i\alpha}^t\}_{\alpha \in
                    I_i^t}}X=k|f_i^t\right)\right),
\end{align}
injects prior belief for each connected component $i$ to actually contain $k$ cells. To this end, a
probabilistic count classifier, $\hat{P}_{\text{count}}$ %\left(\sum_{X \in \{X_{i\alpha}^t\}_{\alpha
% \in I_i^t}}X=k|f_i^t\right)$
, \cf \cref{sec:joint-classifiers}, is trained and applied
on local evidence $f_i^t$ extracted from connected component $i$ at time $t$.
%connects all $N$ regions within the same connected component. The factor penalizes configurations where
%the number of active targets differs from the classifier's suggestion. 
For instance, 
two active regions are favored for both connected components in \cref{fig:joint-pipeline}.
%configurations or more than one active regions are penalized for connected component
%$\{123\}$

\newsavebox{\interframetext}
\savebox{\interframetext}{\small{Inter-Frame Constraints}}
\newlength{\correctLength}
\settowidth{\correctLength}{\usebox{\interframetext}}

\renewcommand{\arraystretch}{1}


The factors above are both associated with variables from single time steps only. To achieve temporal
associations of cells across time steps, the model has to be extended by \emph{inter-frame factors}
which connect detection with transition variables.
%Furthermore, each detection variable is connected to all corresponding transition variables by the outgoing factor
Firstly, \emph{outgoing factors} 
% \begin{align}
%     \label{eq:psi-out}
% \psi_{\mathrm{out}}(X_{i\alpha}^t, &Y^{t,t+1}_{i\alpha ,j_1 \nu_1},...,Y^{t,t+1}_{i\alpha ,j_N \nu_M}) = E_{\text{dis}}(X_{i\alpha}^t, Y^{t,t+1}_{i\alpha ,\bullet})\nonumber \\\nonumber
%     &+E_{\text{div}}(Y^{t,t+1}_{i\alpha , k_1 \mu_1}, Y^{t,t+1}_{i\alpha , k_2 \mu_2}) \\
%     &+E_{\text{trans}}(Y^{t,t+1}_{i\alpha ,\bullet}).
% \end{align}
\begin{align}
    \label{eq:psi-out}
    \psi_{\mathrm{out}}(X_{i\alpha}^t, \mathcal{Y}_{i\alpha\rightarrow}^{t}, f_{i\alpha}^t,
    \Xi_{i\alpha\rightarrow}^{t}) &=
    e^{-E_{\mathrm{out}}(X_{i\alpha}^t, \mathcal{Y}_{i\alpha\rightarrow}^{t}, f_{i\alpha}^t, \Xi_{i\alpha\rightarrow}^{t})} \\
    E_{\mathrm{out}}(X_{i\alpha}^t, \mathcal{Y}_{i\alpha\rightarrow}^{t}, f_{i\alpha}^t,
    \Xi_{i\alpha\rightarrow}^{t}) &= E_{\text{dis}}(X_{i\alpha}^t,
    \mathcal{Y}_{i\alpha\rightarrow}^{t}) \\ \nonumber
    &+ E_{\text{move}}(X_{i\alpha}^t, \mathcal{Y}_{i\alpha\rightarrow}^{t}, f_{i\alpha}^t,
    \Xi_{i\alpha\rightarrow}^{t}) \\ \nonumber
    &+ E_{\text{div}}(X_{i\alpha}^t, \mathcal{Y}_{i\alpha\rightarrow}^{t}, f_{i\alpha}^t,
    \Xi_{i\alpha\rightarrow}^{t}) \\ \nonumber
    &+ E_{\text{out,for}}(X_{i\alpha}^t, \mathcal{Y}_{i\alpha\rightarrow}^{t})
\end{align}
associate each variable $X_{i\alpha}^t$ with all possible transitions
$\mathcal{Y}_{i\alpha\rightarrow}^{t}$ to variables in the successive time step. Here,
$\Xi_{i\alpha\rightarrow}^{t} = \{f_{j\kappa}^{t+1} : \exists Y_{i\alpha,j\kappa}^{t} \in
\mathcal{Y}_{i\alpha\rightarrow}^{t}\}$ is the set of local evidence for all random variables that
are target of one of the transitions in $\mathcal{Y}_{i\alpha\rightarrow}^{t}$.  This factor can be
decomposed into three energy terms: \\ %, each representing a different event:
\begin{enumerate}
      \item Disappearance, \ie termination of a track, is penalized by the \emph{disappearance}
    energy  
    \begin{align}
        \label{eq:energy-dis}
         E_{\mathrm{dis}}(X_{i\alpha}^t,  \mathcal{Y}_{i\alpha\rightarrow}^{t}) &=
        \begin{cases}
            C(X_{i\alpha}^t, \mathcal{Y}_{i\alpha\rightarrow}^{t}), & X_{i\alpha}^t = 1 \wedge
            \sum_{Y\in\mathcal{Y}_{i\alpha\rightarrow}^{t}}Y = 0 \\
            0, & \mathrm{otherwise}
        \end{cases} \\
        C(X_{i\alpha}^t, \mathcal{Y}_{i\alpha\rightarrow}^{t}) &= w_{\mathrm{dis}} \cdot
        \frac{|X_{i\alpha}^t|} {\max_{\kappa \in I_i^t}|X_{i\kappa}^t|} % \\ \nonumber
        % &+ \min\left\{E_{\text{move}}(X_{i\kappa}^t, \mathcal{Y}_{i\kappa\rightarrow}^{t}),
        %     E_{\text{div}}(X_{i\kappa}^t, \mathcal{Y}_{i\kappa\rightarrow}^{t})\right\}_{\kappa \in
        %         I_i^t}.
    \end{align}
    % penalizes disappearance, \ie discontinuity of a track, with the weight
    In other words, cost $w_{\mathrm{dis}}$, weighted by the normalized cardinality of the detection
    variable, is charged when a detection variable is active, but all outgoing transition variables
    are inactive. Note that $\max_{\kappa \in I_i^t}|X_{i\kappa}^t|$ is the cardinality of the
    connected component and is equal to the number of conflict sets within this component. As there
    are cases when targets leave the field of view, disappearances must be possible even though an
    appearance due to non-detection in the following time step is undesirable.
      \item The \emph{division} energy 
    \begin{align}
        \label{eq:energy-div}
        E_{\mathrm{div}}&(X_{i\alpha}^t, \mathcal{Y}_{i\alpha\rightarrow}^{t}, f_{i\alpha}^t,
        \Xi_{i\alpha\rightarrow}^{t}) \\ \nonumber
        &=-w_{\mathrm{div}} \cdot \frac{|X_{i\alpha}^t|} {\max_{\kappa \in I_i^t}|X_{i\kappa}^t|} \\ \nonumber
        &\phantom{=}\times
        \begin{cases}
            \log (F(X_{i\alpha}^t, \mathcal{Y}_{i\alpha\rightarrow}^{t}, f_{i\alpha}^t,
            \Xi_{i\alpha\rightarrow}^{t})), & \sum_{Y\in\mathcal{Y}_{i\alpha\rightarrow}^{t}}Y=2 \\
            0, &\text{otherwise},
        \end{cases}
    \end{align}
    with
    \begin{samepage}
    \begin{alignat}{3}
        F&(X_{i\alpha}^t, \mathcal{Y}_{i\alpha\rightarrow}^{t}, f_{i\alpha}^t,
        \Xi_{i\alpha\rightarrow}^{t}) \\ \nonumber
    %\end{align}\vspace{-1cm}
    %\begin{align*}
        &=\mathlarger{\mathlarger{\sum}}_{\mathsmaller{\substack{Y_{i\alpha,j_1\kappa_1}^t \in \mathcal{Y}_{i\alpha\rightarrow}^{t}, \\
                    Y_{i\alpha,j_2\kappa_2}^t \in \mathcal{Y}_{i\alpha\rightarrow}^{t},\\
                    Y_{i\alpha,j_2\kappa_2}^t \ne Y_{i\alpha,j_1\kappa_1}^t}}}
        \Bigg(\hat{P}_{\mathrm{div}}\left(Y_{i\alpha,j_1\kappa_1}^t=1,Y_{i\alpha,j_2\kappa_2^t}=1|f_{i\alpha}^t,f_{j_1\kappa_1}^{t+1},f_{j_2\kappa_2}^{t+1}\right)
        \\[-35pt] \nonumber
        &\phantom{=}\quad\quad\quad\quad\quad\quad\quad\;\;\;\;\times\mathds{1}(Y_{i\alpha,j_1\kappa_1}^t=1)\mathds{1}(Y_{i\alpha,j_2\kappa_2}^t=1)\Bigg)
    \end{alignat}
    \end{samepage}
    assigns energies to division configurations, \ie two active outgoing transitions, according to
    the prediction of a trained classifier,
    $\hat{P}_{\mathrm{div}}$,
    % \left(Y_{i\alpha,j_1\kappa_1}^t,Y_{i\alpha,j_2\kappa_2^t}|f_{i\alpha}^t,f_{j_1\kappa_1}^{t+1},f_{j_2\kappa_2}^{t+1}\right)$,
    on local evidence $\Xi_{i\alpha\rightarrow}^{t}=\{f_{j\beta}^{t+1} : Y_{j\beta}^t \in
    \mathcal{Y}_{i\alpha\rightarrow}^{t}\}$, and zero energy to any other configurations. Again,
    indicator functions $\mathds{1}(\cdot)$ select the correct transition variables.
      \item In a similar fashion, the \emph{move} energy
    \begin{align}
        E_{\text{move}}&(X_{i\alpha}^t, \mathcal{Y}_{i\alpha\rightarrow}^{t}, f_{i\alpha}^t,
        \Xi_{i\alpha\rightarrow}^{t}) \\ \nonumber
        &=-w_{\text{move}} \cdot \frac{|X_{i\alpha}^t|}{\max_{\kappa \in I_i^t}|X_{i\kappa}^t|} \\
        \nonumber
        &\phantom{=}\times
        \begin{cases}
            \log (G(X_{i\alpha}^t, \mathcal{Y}_{i\alpha\rightarrow}^{t}, f_{i\alpha}^t,
            \Xi_{i\alpha\rightarrow}^{t})), & \sum_{Y\in\mathcal{Y}_{i\alpha\rightarrow}^{t}}Y=1 \\
            0, &\text{otherwise},
        \end{cases}
    \end{align}
    with
    \begin{samepage}
    \begin{align}
        G&(X_{i\alpha}^t, \mathcal{Y}_{i\alpha\rightarrow}^{t}, f_{i\alpha}^t,
        \Xi_{i\alpha\rightarrow}^{t}) \\ \nonumber
    %\end{align}\vspace{-1cm}
    %\begin{align*}
        &=\mathlarger{\mathlarger{\sum}}_{\mathsmaller{\substack{Y_{i\alpha,j\kappa}^t \in \mathcal{Y}_{i\alpha\rightarrow}^{t}}}}
        \Bigg(\hat{P}_{\mathrm{move}}\left(Y_{i\alpha,j\kappa}^t=1|f_{i\alpha}^t,f_{j\kappa}^{t+1}\right)
        \cdot\mathds{1}(Y_{i\alpha,j\kappa}^t=1)\Bigg)
    \end{align}
    \end{samepage}
    is determined by the prediction of a previously trained discriminative classifier,
    $\hat{P}_{\mathrm{move}}$, on % \left(Y_{i\alpha,j\kappa}^t|f_{i\alpha}^t,f_{j\kappa}^{t+1}\right)$, on
    local evidence $\Xi_{i\alpha\rightarrow}^t$. Again, the indicator function $\mathds{1}(\cdot)$
    selects the active transition variable in the summation.
      \item Finally, the \emph{forbidden} energy
    \begin{subnumcases}{E_{\text{out,for}}(X_{i\alpha}^t, \mathcal{Y}_{i\alpha\rightarrow}^{t})}
        \infty, & $\sum_{Y \in \mathcal{Y}_{i\alpha\rightarrow}^{t}}Y > 2$ \\ \label{eq:joint-constraint-out-max}
        \infty, & $\sum_{Y \in \mathcal{Y}_{i\alpha\rightarrow}^{t}}Y > 0 \wedge X_{i\alpha}^t = 0$
        \\ \label{eq:joint-constraint-out-couple}
        0, otherwise
    \end{subnumcases}
    disallows a division of a cell into more than two
    children~(Equation~\ref{eq:joint-constraint-out-max}, $\mathfrak{C}_3$ in
    \cref{tab:constraints}) and couples detection and assignment
    variables~(Equation~\ref{eq:joint-constraint-out-couple}, $\mathfrak{C}_2$ in
    \cref{tab:constraints}): A transition may not be active if one of the associated detections is
    inactive.
\end{enumerate}

The second inter-frame factor, the \emph{incoming factor},

%$ \phi_{\text{in}}(X_{j\beta}^{t+1}, Y^{t}_{\bullet ,j\beta})=$
\begin{align}
    \label{eq:phi-in}
    \psi_{\mathrm{in}}(X_{j\beta}^{t+1}, \mathcal{Y}_{\rightarrow j\beta}^{t})
    &= e^{-E_{\text{in}}(X_{j\beta}^{t+1},  \mathcal{Y}_{\rightarrow j\beta}^{t})} \\
    E_\text{in}(X_{j\beta}^{t+1},  \mathcal{Y}_{\rightarrow j\beta}^{t}) &= E_{\text{app}}(X_{j\beta}^{t+1},
    \mathcal{Y}_{\rightarrow j\beta}^{t}) \\
    &+ E_{\text{in,for}}(X_{j\beta}^{t+1},  \mathcal{Y}_{\rightarrow j\beta}^{t})
\end{align}

is composed of two energy terms:
\begin{enumerate}
      \item The \emph{appearance} energy
    \begin{align}
        E_{\text{app}}&(X_{j\beta}^{t+1}, \mathcal{Y}_{\rightarrow, j\beta}^t) \\ \nonumber
        &=
        \begin{cases}
            \frac{|X_{j\beta}^{t+1}|} {\max_{\kappa \in I_j^{t+1}}|X_{j\kappa}^{t+1}|} \cdot w_{\mathrm{app}}, & X_{j\beta}^{t+1} = 1 \wedge \sum_{Y\in\mathcal{Y}_{\rightarrow, j\beta}^t}Y = 0 \\
            0, & \mathrm{otherwise}
        \end{cases}    
    \end{align}
    assigns the cost $w_{\text{app}}$ in case a cell appears at time $t+1$, \ie $X_{j\beta}$ is active, but none of the
    transition variables in $\mathcal{Y}_{\rightarrow, j\beta}^t$ are, and zero energy to any other configurations.
      \item Similar to the coupling of detection variables and outgoing transition variables, the
    \emph{forbidden} incoming energy
    \begin{subnumcases}{E_{\text{in,for}}(X_{j\beta}^{t+1},  \mathcal{Y}_{\rightarrow j\beta}^{t})=}
        \infty, & $\sum_{Y \in \mathcal{Y}_{\rightarrow j\beta}^{t}}Y >
        1$ \label{eq:joint-constraint-in-max} \\
        \infty, & $\sum_{Y \in \mathcal{Y}_{\rightarrow j\beta}^{t}}Y > 0 \wedge X_{j\beta}^{t+1} =
        0$ \label{eq:joint-constraint-in-couple} \\
        0, & otherwise
    \end{subnumcases}
    couples detections and incoming transitions. Here, since a cell can only have a single, distinct
    ancestor, Equation~\ref{eq:joint-constraint-in-max}~($\mathfrak{C}_5$ in \cref{tab:constraints})
    disallows more than one incoming
    transitions. Equation~\ref{eq:joint-constraint-in-couple}~($\mathfrak{C}_4$ in
    \cref{tab:constraints}) disallows ``dangling'' transitions, \ie a transition always has two
    cells associated with it.
\end{enumerate}
% Just like disappearances appearances are neccessary but unwanted.

With the factors of the factor graph specified, we now define the distribution
\begin{align}
    \label{eq:joint-distribution-fg}
    P(\mathcal{X},\mathcal{Y}) = \frac{1}{Z}\prod_t\prod_i
    \Bigg(&\psi_{\text{count}}(\{X_{i\alpha}^t\}_{\alpha \in I_i^t},f_{i}^t) \\ \nonumber
    &\times\prod_{\alpha}\Big(\psi_{\mathrm{out}}(X_{i\alpha}^t, \mathcal{Y}_{i\alpha\rightarrow}^{t},
    f_{i\alpha}^t, \Xi_{i\alpha\rightarrow}^{t}) \cdot \psi_{\mathrm{in}}(X_{i\alpha}^{t},
    \mathcal{Y}_{\rightarrow i\alpha}^{t-1})\Big) \\ \nonumber
    &\times\prod_{\nu}\Big(\psi_{\text{det}}(\mathcal{K}_{i\nu}^t, \xi_{i\nu}^t)\Big)\Bigg)
\end{align}
over the factor graph, which is a Gibbs distribution
\begin{align}
    P(\mathcal{X},\mathcal{Y}) = \frac{1}{Z}e^{-E(\mathcal{X},\mathcal{Y})}
\end{align}
with energy
\begin{align}
    \label{eq:joint-energy-fg}
    E(\mathcal{X},\mathcal{Y})=\sum_t\sum_i
    \Bigg(&E_{\text{count}}(\{X_{i\alpha}^t\}_{\alpha \in I_i^t},f_{i}^t) \\ \nonumber
    &+\sum_{\alpha}\Big(E_{\mathrm{out}}(X_{i\alpha}^t, \mathcal{Y}_{i\alpha\rightarrow}^{t},
    f_{i\alpha}^t, \Xi_{i\alpha\rightarrow}^{t}) + E_{\mathrm{in}}(X_{i\alpha}^{t},
    \mathcal{Y}_{\rightarrow i\alpha}^{t-1})\Big) \\ \nonumber
    &+\sum_{\nu}\Big(E_{\text{det}}(\mathcal{K}_{i\nu}^t, \xi_{i\nu}^t)\Big)\Bigg).
\end{align}
Parts of such a factor graph are illustrated in \cref{fig:joint-factor_graph}. The factors of
the distribution encourage a meaningful tracking solution and disallow forbidden configurations by
assigning infinite energy. However, for inference,
\begin{align}
    \argmax_{\mathcal{X},\mathcal{Y}}P(\mathcal{X},\mathcal{Y}) = \argmin_{\mathcal{X},\mathcal{Y}}E(\mathcal{X},\mathcal{Y})
\end{align}
we formulate the optimization problem as an integer linear program which allows for direct
incorporation of these constraints by the means of linear constraints. In the following,
\cref{subsec:joint-linear-constraints} describes how the constraints in
Equations~(\ref{eq:joint-factor-segmentation-constraint}), (\ref{eq:joint-constraint-out-max}),
(\ref{eq:joint-constraint-out-couple}), (\ref{eq:joint-constraint-in-max}) and
(\ref{eq:joint-constraint-in-couple}) can be transferred into their respective counterparts
$\mathfrak{C}_1$ to $\mathfrak{C}_5$ in \cref{tab:constraints}, if necessary.


\subsection{Linear Constraints}
\label{subsec:joint-linear-constraints}
\begin{table}
    \scalebox{0.83}{
        \small
        \centering
        \tabcolsep=0.15cm
        \begin{tabularx}{1.2\linewidth}{cXp{4.7cm}p{4.7cm}p{0.5cm}}
            \toprule
            & Constraint Name & Description & Linear Formulation &  ID\\\midrule
            % \parbox[t]{2mm}{\multirow{1}{*}{\rotatebox[origin=c]{90}{\textbf{Intra-Frame}}}} &
            % \multicolumn{2}{l}{Intra-Frame Segmentation Conflicts}&
            Intra-Frame & Segmentation Conflicts&
            Conflicting (\ie overlapping) regions may not be active at the same time. &
            $\displaystyle \mathsmaller{0\le\sum_{\kappa \in\mathcal{C}_{i\alpha}^t}X_{i\kappa}^t \le 1}$ & $\mathfrak{C}_1$\\
            % \parbox[t]{2mm}{\multirow{6}{*}{\rotatebox[origin=c]{90}{Inter-Frame}}}
            \multirow{4}{*}[0.2em]{\hspace{35pt}
                \begin{sideways}\parbox{\correctLength}{\large{Inter-Frame}}\end{sideways}
            }
            & \multicolumn{1}{|l}{Couple-Detection-Outgoing} &
            Inter-frame hypotheses may not be active when the corresponding detection variable is inactive. &
            $\displaystyle \mathsmaller{0 \le X_{i\alpha}^t - Y \le 1 \forall Y \in \mathcal{Y}_{i\alpha\rightarrow}^t}$ & $\mathfrak{C}_2$\\
            & \multicolumn{1}{|l}{Descendants-Outgoing}  &
            A region may not have more than two descendants. &
            $\displaystyle\mathsmaller{ 0 \le \sum_{Y\in\mathcal{Y}_{i\alpha\rightarrow}^t}Y \le 2}$ & $\mathfrak{C}_3$ \\
            & \multicolumn{1}{|l}{Couple-Detection-Incoming} &
            Inter-frame hypotheses may not be active when the corresponding intra-frame hypotheses are inactive. &
            $\displaystyle\mathsmaller{ 0 \le X_{j\beta}^{t+1} - Y \le 1\forall Y \in
                \mathcal{Y}_{\rightarrow j\beta}^t} $ & $\mathfrak{C}_4$\\
            & \multicolumn{1}{|l}{Ancestors-Incoming} &
            A region may not have more than one ancestor. &
            $\displaystyle \mathsmaller{0\le\sum_{Y\in\mathcal{Y}_{\rightarrow j\beta}^t}Y \le 1}$& $\mathfrak{C}_5$\\
            &&&& \\
            \bottomrule
        \end{tabularx}
    }
    \caption{Linear constraints for random variables.}
    \label{tab:constraints}
\end{table}
Many configurations are actually impossible in the cell tracking context.  For instance, a cell
cannot have more than one ascendant or more than two descendants.
% (it can either disappear, move or divide into two children targets). 
Therefore, we add linear constraints to guarantee that only feasible configurations are part
of any solution. Constraints %for detection variables 
within individual time steps are referred to as \emph{intra-frame} constraints
while \emph{inter-frame} constraints regularize the interaction of detection with transition variables.
The constraints are summarized in \cref{tab:constraints} and explained in the following.

Since overlapping -- and hence conflicting -- regions are contained in the segmentation hypotheses,
hard constraints need to restrict the space of feasible solutions to non-contradicting
solutions. For this purpose, Equation~(\ref{eq:joint-factor-segmentation-constraint}) forbids
configurations that contain conflict sets $C_{i\alpha}^t$ with more than one active region, with
linear formulation
\begin{align}
    \mathfrak{C}_1 : 0 \le \sum_{\kappa \in C_{i\alpha}^t}X_{i\kappa}^t \le 1
\end{align}
In \cref{fig:joint-factor_graph}, the red factors and their associated detection variables
represent conflict sets. Taking conflict set $\mathcal{C}_2^1 = \{ X_2^1, X_4^1, X_5^1 \}$ in
\cref{fig:joint-factor_graph} as an example, the constraint states:
\begin{align}
    \mathfrak{C}_1 : 0 \le X_2^1 + X_4^1 + X_5^1 \le 1
\end{align}
Note that the connected component index has been dropped for better readability, in
\cref{fig:joint-factor_graph}. Still, the example is unambiguous, as each time frame only comprises
one connected component and the indices of the regions are unique.

Those intra-frame constraints added, \emph{outgoing} and \emph{incoming} constraints model inter-frame interactions
%other constraints are inter-frame constraints and can be grouped into ``outgoing'' constraints  that
%couple detection variables and assignment variables pointing to the next timestep, and ``incoming''
and couple detection variables with transition variables.
%constraints that couple detection variables and assignment variables coming from the previous
%timestep. In both cases there are constraints 
These constraints
\begin{align}
    \mathfrak{C}_2 : 0 \le X_{i\alpha}^t - Y \le 1 \forall Y \in \mathcal{Y}_{i\alpha\rightarrow}^t
\end{align}
and
\begin{align}
    \mathfrak{C}_4 : 0 \le X_{j\beta}^{t+1} - Y \le 1\forall Y \in \mathcal{Y}_{\rightarrow
        j\beta}^t
\end{align}
in \cref{tab:constraints} ensure compatibility of detection variables and assignment variables: No
transition variable may be active if the corresponding detection variable has state zero.  In terms
of the factor graph in \cref{fig:joint-factor_graph}, this means that, for instance,
% Y_{\{123\}\{23\},\{5\}\{45\}}^{t} \leq X_{\{123\}\{23\}}^t.
\begin{align}
    \mathfrak{C}_2: &0 \le X_5^1 - Y_{5,8}^1 \le 1, \\
    \mathfrak{C}_4: &0 \le X_8^2 - Y_{5,8}^1 \le 1.
\end{align}


In a similar fashion, constraints
\begin{align}
    \mathfrak{C}_3 : 0 \le \sum_{Y\in\mathcal{Y}_{i\alpha\rightarrow}^t}Y \le 2
\end{align}
and
\begin{align}
    \mathfrak{C}_5 : 0\le\sum_{Y\in\mathcal{Y}_{\rightarrow j\beta}^t}Y \le 1
\end{align}
in \cref{tab:constraints} enforce compliance with the tracking requirement that a cell can have at
most two descendants and one ancestor, respectively. In the context of our example, this means, for
instance,
\begin{align}
    \mathfrak{C}_3 : & 0 \le \cdots + Y_{5,8}^1 + \cdots \le 1, \\
    \mathfrak{C}_5 : & 0 \le \cdots + Y_{5,8}^1 + \cdots \le 1.
\end{align}
Here, $\cdots$ indicate potential additional transition variables which are not shown, in order to
keep \cref{fig:joint-factor_graph} clear, but are indicated by dashed lines.

%  only disappear, move, or divide into at most two descendants,
% and can have only zero or one predecessor, \ie merging of two or more cells into a single
% cell is not allowed. Given that $X_{\{123\}\{23\}}^t$, $X_{\{45\}\{4\}}^{t+1}$ and
% $X_{\{45\}\{5\}}^{t+1}$ are active in Fig.~\ref{fig:factor_graph}, the following transitions are possible:
% \begin{itemize}%\itemsep2pt
%       \item $Y_{\{123\}\{23\},{\{45\}\{4\}}}^{t}=0 \wedge Y_{\{123\}\{23\},{\{45\}\{5\}}}^{t}=0$: \\$\{23\}$ disappears, $\{4\}$ and $\{5\}$ appear
%       \item $Y_{\{123\}\{23\},{\{45\}\{4\}}}^{t}=1 \wedge Y_{\{123\}\{23\},{\{45\}\{5\}}}^{t}=0$: \\$\{23\}$ moves to $\{4\}$, $\{5\}$ appears
%       \item $Y_{\{123\}\{23\},{\{45\}\{4\}}}^{t}=0 \wedge Y_{\{123\}\{23\},{\{45\}\{5\}}}^{t}=1$: \\$\{23\}$ moves to $\{5\}$, $\{4\}$ appears
%       \item $Y_{\{123\}\{23\},{\{45\}\{4\}}}^{t}=1 \wedge Y_{\{123\}\{23\},{\{45\}\{5\}}}^{t}=1$: \\$\{23\}$ divides to $\{4\}$, $\{5\}$ 
% \end{itemize}

A feasible tracking solution must fulfill all constraints $\mathfrak{C}_1$-$\mathfrak{C}_5$. 
It should be noted that only $\mathfrak{C}_3$ needs to be adjusted appropriately if non-divisible objects 
are to be tracked, \ie
\begin{align}
    \mathfrak{C}_3^{\prime} : 0 \le \sum_{Y\in\mathcal{Y}_{i\alpha\rightarrow}^t}Y \le 1.
\end{align}

With all constraints in linear formulation, we can now formulate MAP inference, $\argmin
E(\mathcal{X},\mathcal{Y})$, as an integer linear program~(\cref{cha:app-ilp}):
\begin{align}
    &\argmin_{\mathcal{X},\mathcal{Y}} E(\mathcal{X}, \mathcal{Y}) \\\nonumber
    &\;\;\;\;\;\; \st \\ \nonumber
    &\mathfrak{C}_1,\mathfrak{C}_2,\mathfrak{C}_3,\mathfrak{C}_4,\mathfrak{C}_5
\end{align}
With \cref{eq:joint-energy-fg} and constraints $\mathfrak{C}_1$ to $\mathfrak{C}_5$ written out, this
becomes:
\begin{samepage}
\begin{flalign}
    \phantom{=}&\argmin_{\mathcal{X},\mathcal{Y}} E(\mathcal{X},\mathcal{Y})  \\ \nonumber
    =&\argmin_{\mathcal{X},\mathcal{Y}} \sum_t\sum_i
    \Bigg(E_{\text{count}}(\{X_{i\alpha}^t\}_{\alpha \in I_i^t},f_{i}^t)
    +\sum_{\alpha}\Big(E_{\mathrm{out}}(X_{i\alpha}^t, \mathcal{Y}_{i\alpha\rightarrow}^{t},
    f_{i\alpha}^t, \Xi_{i\alpha\rightarrow}^{t}) \\ \nonumber
    &\phantom{\argmin_{\mathcal{X},\mathcal{Y}} \sum_t\sum_i
    \Bigg(}+ E_{\mathrm{in}}(X_{i\alpha}^{t}, \mathcal{Y}_{\rightarrow i\alpha}^{t-1})\Big)+\sum_{\nu}\Big(E_{\text{det}}(\mathcal{K}_{i\nu}^t, \xi_{i\nu}^t)\Big)\Bigg)
\end{flalign}
\begin{align*}
    &\;\;\;\;\;\;\st \\
    &\mathfrak{C_1} : 0\le\sum_{\kappa \in\mathcal{C}_{i\alpha}^t}X_{i\kappa}^t \le 1 \; \forall \; i,t,\alpha\\
    &\mathfrak{C_2} : 0 \le X_{i\alpha}^t - Y \le 1 \;\forall\; Y \in \mathcal{Y}_{i\alpha\rightarrow}^t
    \; \forall \; i,t,\alpha \\
    &\mathfrak{C_3} : 0 \le \sum_{Y\in\mathcal{Y}_{i\alpha\rightarrow}^t}Y \le 2 \;\forall\; i,t,\alpha \\
    &\mathfrak{C_4} : 0 \le X_{i\alpha}^{t} - Y \le 1\;\forall\; Y \in\mathcal{Y}_{\rightarrow i\alpha}^{t-1}
    \; \forall i,t,\alpha \\
    &\mathfrak{C_5} : 0\le\sum_{Y\in\mathcal{Y}_{\rightarrow i\alpha}^{t-1}}Y \le 1 \;\forall\; i,t,\alpha
\end{align*}
\end{samepage}
CPLEX is used for solving the integer linear program.
In the following, we will introduce the classifiers.

%%% Local Variables: 
%%% mode: latex
%%% TeX-master: "../../../main"
%%% End: 
