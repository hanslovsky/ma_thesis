\section{Oversegmentation}
\label{sec:joint-oversegmentation}

The oversegmentation in the joint tracking and segmentation model follows a three-step procedure:
\begin{enumerate}
    \item Obtain a partition into foreground and background such that as many cells as possible are
  included -- not necessarily as single objects -- in the foreground~(\cref{subsec:joint-undersegmentation}).
    \item Partition the foreground mask into segments with the aim of an oversegmentation of all
  contained cells~(\cref{subsec:joint-oversegmentation}).
    \item Form a richer set of segmentation hypotheses by gradually merging segments into larger
  regions~(\cref{subsec:joint-region-merging}).
\end{enumerate}
In the following, we give a detailed description of our approach towards each of these tasks and the
generation of the hypotheses graph. Note that, in general, it is possible to exchange our approach
with a suitable replacement at any point of the three-step procedure.

\subsection{Initial Partitioning}
\label{subsec:joint-undersegmentation}
In general, the initial partitioning of the data into foreground and background follows the
procedure of the segmentation in the chain graph~(\cref{subsec:fg-chaingraph}) and
conservation~(\cref{subsec:fg-conservation}) tracking methods. First, a random forest
classifier~(\cref{cha:app-rf}) is trained on a small number of user annotated training samples using
ILASTIK~\citep{sommer_11_ilastik}. Then, the classifier predicts \emph{probability maps} for the
complete data set which can be transferred into a partitioning into foreground and background by
thresholding on the foreground probability of each pixel.

Contrary to the previous methods, chain graph and conservation tracking, the initial partitioning is
only the basis for another segmentation and thus we do not aim for a segmentation that has a
bijective mapping from cells in the raw data to foreground connected components. In fact, the
foreground includes as many cells as possible, even if they are merged into large connected
components.

\subsection{Oversegmentation}
\label{subsec:joint-oversegmentation}
Now an oversegmentation of the foreground mask is generated. To this end, the watershed
algorithm~\citep{vincent_91_watersheds,roerdink_00_watershed} is applied to the inverse of
the distance transform~\citep[Section~18.4.4]{jaehne_05_digital} of the foreground regions with seeds
at the local maxima of the distance transform, \ie the points on the foreground mask which are
locally furthest away from any background region. This gives rise to regularly shaped compact
segments, potential cell segmentations, which are the nodes of the initial region adjacency
graph. For each pair of conterminous regions, an edge between the corresponding nodes is added to
the graph.

\subsection{Region Merging}
\label{subsec:joint-region-merging}
Starting from the initial region adjacency graph, regions are merged into a richer set of possibly
competing cell segmentation hypotheses. For simplicity, we choose a hierarchical region merging with
$L$ levels. However, any region merging algorithm is applicable. The weight $w$ of an edge $E$
between two nodes $r_1$, $r_2$ in the region adjacency graph is a similarity measure for neighboring
regions based on the joint border of the involved regions. At each level $1 \le l \le L$ all pairs of
regions with similarity greater than a threshold $\tau$,
\begin{align}
    w(r_1, r_2) \ge \tau,
\end{align}
are merged into a new region. In the region adjacency graph, the two corresponding nodes are
contracted.

% \begin{figure}
%     \centering
%     \begin{subfigure}{0.48\textwidth}
%         \centering
%         \begin{tikzpicture}[baseline=(image1.north)]
    \node[anchor=south west,inner sep=0] (image1) {
        \includegraphics[width=0.7\textwidth]{images/joint/overseg/75/02/colored00.png}};
    \begin{scope}[x={(image1.south east)},y={(image1.north west)}]
        \node[region_id] (r1) at (0.4, 0.75) {\huge{$1$}};
        \node[region_id] (r2) at (0.65, 0.7) {\huge{$2$}};
        \node[region_id] (r3) at (0.5, 0.32) {\huge{$3$}};
        \node[rectangle, draw, fit=(r1) (r2), color=green,line width=2pt, inner sep=8mm] (f) {};
        % helpers
        % http://tex.stackexchange.com/questions/9559/drawing-on-an-image-with-tikz/9561
        % \draw[help lines,xstep=.1,ystep=.1] (0,0) grid (1,1);
        % \foreach \x in {0,2,...,8} { \node [anchor=north] at (\x/10,0) {0.\x}; }
        % \foreach \y in {0,2,...,8} { \node [anchor=east] at (0,\y/10) {0.\y}; }
    \end{scope}
\end{tikzpicture}


%%% Local Variables: 
%%% mode: latex
%%% TeX-master: "../../../main"
%%% End: 

%         \caption{Regions $1$ and $2$ are similar and therefore merging is preferable.}
%     \end{subfigure}
%     \hfill
%     \begin{subfigure}{0.48\textwidth}
%         \centering
%         \begin{tikzpicture}[baseline=(image2.north)]
    \node[anchor=south west,inner sep=0] (image2) {
        \includegraphics[width=0.7\textwidth]{images/joint/overseg/75/02/colored01_all.png}};
    \begin{scope}[x={(image1.south east)},y={(image2.north west)}]
        \node[region_id] (r4) at (0.53, 0.73) {\huge{$4$}};
        \node[region_id] (r3) at (0.5, 0.32) {\huge{$3$}};
        \node[rectangle, draw, fit=(r3) (r4), color=red,line width=2pt, inner sep=9.3mm, text
        width=5em, yshift=-0.7mm] (f) {};
    \end{scope}
\end{tikzpicture}

%%% Local Variables: 
%%% mode: latex
%%% TeX-master: "../../../main"
%%% End: 

%         \caption{The union of regions $3$ and $4$ is unlikely to be a cell. Therefore, the
%             similarity is low.}
%     \end{subfigure}
%     \caption[Different edge weights]{Edge weights between different pairs of region indicated by the
%     color of the circumscribing rectangle: green and red indicate high and low similarity respectively.}
%     \label{fig:joint-overseg-edge-weight}
% \end{figure}

By construction, regions potentially contradict each other. These conflicts are graphically
represented by the conflict graph $G_{\text{con}}$ with a node for each region and an edge for each
pair of regions which share at least one common segment:
\begin{align}
    G_{\text{con}} &= (\mathcal{V}_{\text{con}}, \mathcal{E}_{\text{con}}), \\
    \mathcal{E}_{\text{con}} &= \{V_1 \undirected V_2  : r_{V_1} \cap r_{V_2} \ne \emptyset\}.
\end{align}

\subsection{Hypotheses Graph}
\label{subsec:joint-hypotheses-graph}
In contrast to the chain graph method, the hypotheses graph of the joint segmentation and tracking
does not contain single detection and transition hypotheses. In fact, each node represents a
connected component and thus contains as many segmentation hypotheses as there are regions within
that connected component. In a similar fashion, the edges represent all possible transitions between
regions in the respective connected components.  Likewise to the chain graph hypotheses graph, edges
between connected components are determined by a nearest neighbor search. In practice, however, a
restriction on the number of possible transitions advisable in order to reduce the model size. To
this end, assignment hypotheses are generated only between all regions in each conflict set in a
connected component $c^t$ at time $t$ and all regions in the nearest neighbor conflict set for each
connected component $c^{t+1}$ at time $t+1$ with an edge $c^t \undirected c^{t+1}$ in the hypotheses
graph. In this context, the distance between two conflict sets is measured by the Euclidean distance
between the two segments that define these conflict sets. As a result of this \emph{pruning},
regions that are further up in the segmentation hierarchy have more potential assignments, with the
outgoing assignments of a connected component being not affected by the pruning. Finally, a factor
graph is constructed from the hypotheses graph.

% Before joint segmentation and tracking becomes computationally feasible, 
% the time-series of 2D/3D images/volumes need to be preprocessed
% to reduce the problem space (step (II) and (III) in \cref{fig:pipeline}). 
% Here, the images are processed each independently which allows for
% paralellization of the oversegmentation algorithm. 
% In step (II), the purpose is to obtain an oversegmentation on every image 
% which is sufficiently fine but as coarse as possible.
% That is, we prefer single segments (superpixels) for (isolated) objects without
% ambiguities, whereas multiple (smaller) segments are desired in cases where objects overlap in space. 

% To this end, we propose the following oversegmentation algorithm:
% \begin{enumerate}
%  \item Obtain a coarse segmentation which only distinghuishes 
% potential foreground from certain background.
%  \item Automatically select seeds fulfilling the requirements outlined above.
%  \item Compute the seeded watershed on the foreground mask.
%  \item Merge resulting segments hierarchically to potential regions.
% \end{enumerate}

% Here, the first step may be performed by any segmentation algorithm which can be adjusted in a way that 
% only those pixels/voxels are predicted as background where we are most sure. This step's output
% is either a hard segmentation or a probability map of the foreground (soft segmentation). 
% Note that typically, it is not desirable to track the resulting connected components directly, since 
% large clusters of cells may be contained in each connected component. Hence, we continue by 
% refining these connected components to multiple segments.

% To this end, the watershed algorithm is applied to the mask of potential foreground (``\emph{foreground mask}'').
% The seeds for it are determined by the local maxima of the distance transform on the foreground mask
% to nearest background pixels/voxels. This gives rise to regularly shaped compact segments, potential cell segmentations. 
% A minimum
% size of these segments may be achieved by performing a dilation operation on the seeds with appropriate disc radius.
% Note that the watershed may either be applied to the \emph{soft} segmentation mask or the masked raw data directly.

% Finally, segments are merged to regions, possible competing cell segmentations (step (III) in \cref{fig:pipeline}).  
% This merging can be performed in any order, but
% for simplicity, we choose a hierarchical region merging in a region adjacency graph using $L$ levels. 
% Its edge weights between neighboring segments/regions may be arbitrary complex and the regions may be merged 
% in an order determined by these edge weights.

% Since the (merged) regions will spatially overlap with their original segments, natural conflicts between 
% these regions exist and are incorporated into our graphical model (step (IV) in \cref{fig:pipeline})
% as discussed in the next section.


%%% Local Variables: 
%%% mode: latex
%%% TeX-master: "../../../main"
%%% End: 
