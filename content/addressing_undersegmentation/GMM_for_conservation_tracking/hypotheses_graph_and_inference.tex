\section{Hypotheses Graph on Subset of Cells and Inference}
\label{sec:gmm-hypotheses}
The conservation tracking constructs a hypotheses graph (\cref{fig:gmm-hypotheses-graph-orig}) of
possible cell transitions (edges) between detections (nodes) in consecutive time frames. It then
formulates the assignment problem as a factor graph with random variables for detections and
assignments (see \cref{subsec:fg-conservation} for more details). Finally, the MAP solution on the
factor graph yields the correct (active) assignments and detections. In the modified hypotheses
graph, detection variables now take values between zero (inactive) and the maximum number of allowed
mergers per connected component. According to the inferred MAP configuration, the assignment
variables are labeled active ($\ge 1$) or inactive ($0$). This modified subgraph
(\cref{fig:gmm-hypotheses-graph-inferred}, inactive edges have been removed for visualization) is
the basis for cell reconstruction and merger resolution by the means of GMMs. In general, the merger
resolution follows these steps:
\begin{enumerate}
      \item Extract a subset of the hypotheses graph containing
    \begin{itemize}
          \item all detection nodes with $X_i^t = k > 1$, \ie all merged objects,
          \item all active transition edges connected to these detection variables, and
          \item all active detection nodes $X_i^t = \bar{k} = 1$ that are connected to a merger
        detection node by an active transition edge.
    \end{itemize}
      \item Infer $k$ means and covariances for all merger detection nodes by following
    \cref{sec:gmm-gmm}, \ie fit a GMM clustering to each connected component associated with a merger
    object separately.
      \item Replace each merger detection node by $k$ new detection nodes and duplicate the
    assignment variables analogously.
      \item Formulate the factor graph and rerun inference on the subgraph with new constraints to
    determine true assignments.
\end{enumerate}
In the following, each of these steps is defined in detail by means of a hypothetical example with
figures that illustrate the procedure. Despite the example being hypothetical, it still reflects a
hypotheses graph that may occur in a real tracking problem as part of a larger hypotheses graph.

\cref{fig:gmm-hypotheses-graph-orig} shows an exemplary hypotheses graph over four time steps. This
hypotheses graph is constructed from connected components on the segmented image.  Here, each node
is assumed to refer to a connected component containing at least one real cell, \ie no false
positives. The transition hypotheses are created based on a thresholded one-nearest-neighbor search,
both in forward and backward direction in time, where the distance between two cells $c_1$, $c_2$
\begin{align}
    \label{eq:gmm-region-distance}
    \dis(c_1, c_2) &= \dis\left(\rc(c_1), \rc(c_2)\right) = \lVert \rc(c_1) - \rc(c_2) \rVert_2^2 \\
    \label{eq:gmm-region-center}
    \rc_i(c) &= \frac{1}{\lvert c \rvert} \sum_{p \in c} p_i \; \forall i\in\{1,\hdots,N\}
\end{align}
is the squared $L_2$-distance between their region centers $\rc(c_1)$, $\rc(c_2)$. The $i$-th
component of the region center of cell $c$ in $N$ spatial dimensions is calculated in
\cref{eq:gmm-region-center}. $p_i$ is the $i$-th coordinate of pixel $p$ in an $N$-dimensional volume that is part
of cell $c$ that, in turn, consists of $\lvert c \rvert$ pixels.

The only ``non-competing'' transitions are $T_{1\to 5}^t$ connecting $X_1^t$ to $X_5^{t+1}$ and
$T_{8\to 10}^{t+2}$ from $X_8^{t+2}$ to $X_{10}^{t+3}$ that can be identified as moves directly from
the original hypotheses graph. For all other transitions, inference on the factor graph yields an
optimal solution which best resembles a meaningful tracking. The factor graph construction is
explained in \cref{subsec:fg-conservation}.


\begin{figure}[h]
    \centering
    \scalebox{0.7}{
        \begin{tikzpicture}[minimum size=58pt,scale=0.45, every node/.style={scale=0.45, font=\LARGE}, thick]
            \input{images/gmm/gmm_hyp_graph_subset/hypotheses_original.tex}
        \end{tikzpicture}
    }
    %\rule{\textwidth}{0.3pt}
    \caption[Original Hypotheses Graph]{Explanatory example of a hypotheses graph that is
        constructed as the first step of Conservation Tracking. Note that edges are only potential
        assignments and nodes can represent false positive detection. Thus, the states of all nodes
        and edges -- active ($\ge 1$) or inactive ($0$) -- are inferred by the conservation tracking
        factor graph. }
    \label{fig:gmm-hypotheses-graph-orig}
\end{figure}

% saveboxes for node descriptions
\newsavebox{\captionA} \savebox{\captionA}{\small\tikz[baseline, inner sep=1pt]\node[thick, draw,
    circle, hypotheses_one_object, text=black] at (0,2.5pt){\tiny $X$};} \newsavebox{\captionB}
\savebox{\captionB}{\small\tikz[baseline, inner sep=1pt]\node[thick, draw, circle,
    hypotheses_two_objects, text=black] at (0,2.5pt){\tiny $X$};} \newsavebox{\captionC}
\savebox{\captionC}{\small\tikz[baseline, inner sep=1pt]\node[thick, draw, circle,
    hypotheses_three_objects, text=black] at (0,2.5pt){\tiny $X$};} \newsavebox{\captionD}
\savebox{\captionD}{\small\tikz[baseline, inner sep=1pt]\node[thick, draw, circle,
    text=black, hypotheses_new_detection] at (0,2.5pt){\tiny $X$};}
\newsavebox{\captionE} \savebox{\captionE}{\small\tikz[baseline, inner sep=1pt]\path[draw,
    hypotheses_new_transition, very thick, dotted] (0,2.5pt) edge (10pt, 2.5pt);} After inference on
the factor graph yielded a globally optimal solution, inactive edges and nodes are removed from the
hypotheses graph and. Furthermore, nodes reflect the state of the corresponding random variables. In
the presented example, this means that $T_{3\to 7}^t$ and $T_{5\to 9}^{t+1}$ are pruned in
\cref{fig:gmm-hypotheses-graph-inferred} and nodes are color coded according to their state. This is
the intermediate tracking result for the conservation tracking.
\begin{figure}
    \centering
    \scalebox{0.7}{
        \begin{tikzpicture}[minimum size=58pt,scale=0.45, every node/.style={scale=0.45, text=black, font=\LARGE}, thick]
            \begin{scope}
    \node (t1) {\huge $t$};
    \node[color=hypothesesoneobject, hypotheses_one_object, below=of t1, circle, draw] (x11) {$X_1^t$};
    \node[color=hypothesesoneobject, hypotheses_one_object, below=of x11, circle, draw] (x12) {$X_2^t$};
    \node[color=hypothesesoneobject, hypotheses_one_object, below=of x12, circle, draw] (x13) {$X_3^t$};
    \node[color=hypothesesoneobject, hypotheses_one_object, below=of x13, circle, draw] (x14) {$X_4^t$};
\end{scope}


\begin{scope}
    \node[right=of t1, xshift=15mm] (t2) {\huge $t+1$};
    \node[color=hypothesesoneobject, hypotheses_one_object, below=of t2, circle, draw] (x21) {$X_5^{t+1}$};
    \node[color=hypothesesoneobject, hypotheses_one_object, right=of x14, circle, draw, xshift=15mm] (x23) {$X_7^{t+1}$};
    \node[color=hypothesestwoobjects, hypotheses_two_objects, circle, draw] (x22) at ($(x21)!0.5!(x23)$) {$X_6^{t+1}$};
\end{scope}

\begin{scope}
    \node[right=of t2, xshift=15mm] (t3) {\huge $t+2$};
    \node[color=hypothesesoneobject, hypotheses_one_object, below=of t3, circle, draw] (x31) {$X_8^{t+2}$};
    \node[color=hypothesesthreeobjects, hypotheses_three_objects, circle, draw, shift=($(x31.center)-(x21.center)$)] (x32) at (x22) {$X_9^{t+2}$};
\end{scope}

\begin{scope}
    \node[right=of t3, xshift=15mm] (t4) {\huge $t+3$};
    \node[color=hypothesesoneobject, hypotheses_one_object, below=of t4, circle, draw] (x41) {$X_{10}^{t+3}$};
    \node[color=hypothesesoneobject, hypotheses_one_object, below=of x41, circle, draw] (x42) {$X_{11}^{t+3}$};
    \node[color=hypothesesoneobject, hypotheses_one_object, below=of x42, circle, draw] (x43) {$X_{12}^{t+3}$};
    \node[color=hypothesesoneobject, hypotheses_one_object, below=of x43, circle, draw] (x44) {$X_{13}^{t+3}$};
\end{scope}

\begin{scope}[on background layer]
    \node[rectangle, draw, color=hypothesesbackground!40, fill=hypothesesbackground!30,
    fit=(x11) (x12) (x13) (x14), inner sep=13mm] (b1) {};
    \node[rectangle, draw, color=hypothesesbackground!40, fill=hypothesesbackground!30,
    fit=(x21) (x22) (x23), inner sep=13mm] (b2) {};
    \node[rectangle, draw, color=hypothesesbackground!40, fill=hypothesesbackground!30,
    fit=(b2), shift=($(t3.center) - (t2.center)$), inner sep=-0.1mm] (b3) {};
    \node[rectangle, draw, color=hypothesesbackground!40, fill=hypothesesbackground!30,
    fit=(x41) (x42) (x43) (x44), inner sep=13mm] (b4) {};
\end{scope}

\path[hypothesestransition] (x11) edge (x21);
\path[hypothesestransition] (x12) edge (x22);
\path[hypothesestransition] (x13) edge (x22);
\path[hypothesestransition] (x14) edge (x23);

\path[hypothesestransition] (x21) edge (x31);
\path[hypothesestransition] (x22) edge (x32);
\path[hypothesestransition] (x23) edge (x32);

\path[hypothesestransition] (x31) edge (x41);
\path[hypothesestransition] (x32) edge (x42);
\path[hypothesestransition] (x32) edge (x43);
\path[hypothesestransition] (x32) edge (x44);

%%% Local Variables: 
%%% mode: latex
%%% TeX-master: "../../../main"
%%% End: 

        \end{tikzpicture}
    }
    %\rule{\textwidth}{0.3pt}
    \caption[Hypotheses Graph with inferred result]{Hypotheses Graph after Inference: The colors
        indicate the number of objects  in the connected component: one object
        $\left(\protect\usebox{\captionA}\right)$, two objects $\left(\protect\usebox{\captionB}\right)$ or three
        objects $\left(\protect\usebox{\captionC}\right)$. Inactive edges have been removed from the
        original graph in \cref{fig:gmm-hypotheses-graph-orig}.}
    \label{fig:gmm-hypotheses-graph-inferred}
\end{figure}

For the further procedure, a subgraph of the hypotheses graph is formed by taking only
those nodes into account that represent merged objects, \usebox{\captionB} and \usebox{\captionC} in
our example, or share an edge with such a node. All edges connected to a merger node are part of the
subgraph as well. From \cref{fig:gmm-hypotheses-graph-inferred}, the subgraph
\begin{align}
    \label{eq:gmm-subgraph}
    G_{\text{sub}} &= \left(\mathcal{X}_{\text{sub}}, \mathcal{T}_{\text{sub}}\right), \\
    \mathcal{X}_{\text{sub}} &=
    \left\{X_2^t, X_3^t, X_6^{t+1}, X_7^{t+1}, X_9^{t+2}, X_{11}^{t+3}, X_{12}^{t+3},
        X_{13}^{t+3}\right\} \\
    \mathcal{T}_{\text{sub}} &= \left\{T_{2\to 6}^t, T_{3\to 6}^t, T_{6\to 9}^{t+1}, T_{7\to
            9}^{t+1}, T_{9\to 11}^{t+2}, T_{9\to 12}^{t+2}, T_{9\to 13}^{t+2} \right\}
\end{align}
is extracted.

Now, $G_{\text{sub}}$ is modified with the aim that each node in the graph represents a single cell,
\ie no merged objects remain in the graph in \cref{fig:gmm-hypotheses-graph-subset}. This is where
the application of Gaussian mixture models comes into play: First, for each merger node, a GMM is
fit according to the inferred number of objects $k$, with the means $\left\{\MUH\right\}_{h=1\hdots
    k}$ estimating the region centers of the individual cells comprised in the connected component
represented by that node. Then, the merger node is replaced by $k$ new nodes
$\left(\usebox{\captionD}\right)$, one for each of the deduced cells.

In addition, contrary to merger nodes, which may not be dividing objects, a division of a single cell
can occur at one time step previous to a merging event. In that case and since the constraints on
the subgraph disallow divisions in the subgraph (\cref{eq:no-division-constraint}), if at least one
of the two children cells is part of a merged object, the parent node needs to be duplicated
(\cref{fig:gmm-division-duplication}). These identical and interchangeable duplicates are rejoined
after inference. Specifically, the duplicates guarantee the validity of the formulation of the
constraints (\crefrange{eq:main}{eq:cons-sub-in-e}) for inference on the subgraph in the presence of
divisions with child cells that are part of merged objects.

Finally, edges %$\left(\usebox{\captionE}\right)$
(dotted lines) are added according to the edges formerly connected to the merger nodes. For each of
these original edges, $k$ new edges are introduced, connecting a new node to the corresponding
counterpart of the old edge. After discarding the merger nodes and associated edges, the hypotheses
graph looks like \cref{fig:gmm-hypotheses-graph-subset}.

\begin{figure}
    \centering
    \scalebox{0.7}{
        \begin{tikzpicture}[minimum size=58pt,scale=0.45, every node/.style={scale=0.45, text=black, font=\LARGE}, thick]
            
\begin{scope}
    \node (t1) {\huge $t$};
    \node[hypothesesdetection, below=of t1, circle, draw] (x11) {$X_2^t$};
    \node[hypothesesdetection, below=of x11, circle, draw] (x12) {$X_3^t$};
\end{scope}


\begin{scope}
    \node[right=of t1, xshift=15mm] (t2) {\huge $t+1$};
    \node[hypotheses_new_detection, below=of t2, circle, draw] (x21) {$X_{14}^{t+1}$};
    \node[hypotheses_new_detection, below=of x21, circle, draw] (x22) {$X_{15}^{t+1}$};
    \node[hypothesesdetection, below=of x22, circle, draw] (x23) {$X_{7}^{t+1}$};
\end{scope}

\begin{scope}
    \node[right=of t2, xshift=15mm] (t3) {\huge $t+2$};
    \node[hypotheses_new_detection, below=of t3, circle, draw] (x31) {$X_{16}^{t+2}$};
    \node[hypotheses_new_detection, below=of x31, circle, draw] (x32) {$X_{17}^{t+2}$};
    \node[hypotheses_new_detection, below=of x32, circle, draw] (x33) {$X_{18}^{t+2}$};
\end{scope}

\begin{scope}
    \node[right=of t3, xshift=15mm] (t4) {\huge $t+3$};
    \node[hypothesesdetection, below=of t4, circle, draw] (x41) {$X_{11}^{t+3}$};
    \node[hypothesesdetection, below=of x41, circle, draw] (x42) {$X_{12}^{t+3}$};
    \node[hypothesesdetection, below=of x42, circle, draw] (x43) {$X_{13}^{t+3}$};
\end{scope}

\begin{scope}[on background layer]
    \node[rectangle, draw, color=hypothesesbackground!40, fill=hypothesesbackground!30,
    fit=(x11) (x12) (x13), inner sep=13mm] (b1) {};
    \node[rectangle, draw, color=hypothesesbackground!40, fill=hypothesesbackground!30,
    fit=(x21) (x22) (x23), inner sep=13mm] (b2) {};
    \node[rectangle, draw, color=hypothesesbackground!40, fill=hypothesesbackground!30,
    fit=(x31) (x32) (x33), inner sep=13mm] (b3) {};
    \node[rectangle, draw, color=hypothesesbackground!40, fill=hypothesesbackground!30,
    fit=(x41) (x42) (x43), inner sep=13mm] (b4) {};
\end{scope}

\path[hypotheses_new_transition] (x11) edge (x21);
\path[hypotheses_new_transition] (x11) edge (x22);


\path[hypotheses_new_transition] (x12) edge (x21);
\path[hypotheses_new_transition] (x12) edge (x22);

\path[hypotheses_new_transition] (x21) edge (x31);
\path[hypotheses_new_transition] (x21) edge (x32);
\path[hypotheses_new_transition] (x21) edge (x33);

\path[hypotheses_new_transition] (x22) edge (x31);
\path[hypotheses_new_transition] (x22) edge (x32);
\path[hypotheses_new_transition] (x22) edge (x33);

\path[hypotheses_new_transition] (x23) edge (x31);
\path[hypotheses_new_transition] (x23) edge (x32);
\path[hypotheses_new_transition] (x23) edge (x33);

\path[hypotheses_new_transition] (x31) edge (x41);
\path[hypotheses_new_transition] (x31) edge (x42);
\path[hypotheses_new_transition] (x31) edge (x43);

\path[hypotheses_new_transition] (x32) edge (x41);
\path[hypotheses_new_transition] (x32) edge (x42);
\path[hypotheses_new_transition] (x32) edge (x43);

\path[hypotheses_new_transition] (x33) edge (x41);
\path[hypotheses_new_transition] (x33) edge (x42);
\path[hypotheses_new_transition] (x33) edge (x43);

%%% Local Variables: 
%%% mode: latex
%%% TeX-master: "../../../main"
%%% End: 

        \end{tikzpicture}
    }
    %\rule{\textwidth}{0.3pt}
    \caption[Subset of hypotheses graph with newly created nodes]{Subset of hypotheses graph with
        newly created nodes $\left(\protect\usebox{\captionD}\right)$ and edges
        $\left(\protect\usebox{\captionE}\right)$. }
    \label{fig:gmm-hypotheses-graph-subset}
\end{figure}

\newsavebox{\captionF}
\savebox{\captionF}{\small\tikz[baseline, inner sep=1pt]\node[thick, draw, circle,
    hypotheses_division_duplicate, text=black] at (0,2.5pt){\tiny $X$};}

\begin{figure}
    \centering
    % \begin{tabular}{c|c|c}
    \begin{subfigure}[t]{0.32\textwidth}
        \centering
        \begin{tikzpicture}[minimum size=58pt,scale=0.35, every node/.style={scale=0.35, text=black, font=\LARGE}, thick]
                \begin{scope}
        \node (t1) {\huge $t$};
        \node[hypotheses_one_object, below=of t1, circle, draw] (x11) {$X_1^t$};
        \node[hypotheses_one_object, below=of x11, circle, draw] (x12) {$X_2^t$};
    \end{scope}
    \begin{scope}[on background layer]
        \node[hypotheses_division_duplicate, below=of x12, circle, draw, dashed] (x13) {$\bar{X}_2^t$};
    \end{scope}
    
    
    \begin{scope}
        \node[right=of t1, xshift=-6mm] (t2) {\huge $t+1$};
        \node[hypotheses_two_objects, below=of t2, circle, draw] (x21) {$X_3^{t+1}$};
        \node[hypotheses_one_object, below=of x21, circle, draw] (x22) {$X_{4}^{t+1}$};
    \end{scope}
    \begin{scope}[on background layer]
        \node[hypothesesdetection, below=of x22, circle, draw] (x23)  {$X_4^{t+1}$};
    \end{scope}
    
    \begin{scope}
        \node[right=of t2, xshift=-6mm] (t3) {\huge $t+2$};
        \node[hypotheses_one_object, below=of t3, circle, draw] (x31) {$X_5^{t+2}$};
        \node[hypotheses_one_object, below=of x31, circle, draw] (x32) {$X_6^{t+2}$};
        \node[hypotheses_one_object, below=of x32, circle, draw] (x33) {$X_7^{t+2}$};
    \end{scope}
    

    \begin{scope}[on background layer]
        \node[rectangle, draw, color=hypothesesbackground!40, fill=hypothesesbackground!30,
        fit=(x11) (x12) (x13), inner sep=6mm] (b1) {};
        \node[rectangle, draw, color=hypothesesbackground!40, fill=hypothesesbackground!30,
        fit=(x21) (x22) (x23), inner sep=6mm] (b2) {};
        \node[rectangle, draw, color=hypothesesbackground!40, fill=hypothesesbackground!30,
        fit=(x31) (x32) (x33), inner sep=6mm] (b3) {};
    \end{scope}

    \path[hypothesestransition] (x11) edge (x21);
    \path[hypothesestransition] (x12) edge (x21);
    \path[hypothesestransition] (x12) edge (x22);

    \path[hypothesestransition] (x21) edge (x31);
    \path[hypothesestransition] (x21) edge (x32);
    \path[hypothesestransition] (x22) edge (x33);


%%% Local Variables: 
%%% mode: latex
%%% TeX-master: "../../../main"
%%% End: 

        \end{tikzpicture}
        \caption{Division $X_2^t$ at time $t$ with a child that is part of merger $X_3^{t+1}$ at
            time $t+1$. As before, $\left(\protect\usebox{\captionA}\right)$ and
            $\left(\protect\usebox{\captionB}\right)$ denote one and two objects respectively.}
        \label{subfig:gmm-division-into-merger}
    \end{subfigure}
    \hfill
    \begin{subfigure}[t]{0.32\textwidth}
        \centering
        \begin{tikzpicture}[minimum size=58pt,scale=0.35, every node/.style={scale=0.35, text=black, font=\LARGE}, thick]
            \input{images/gmm/gmm_hyp_graph_subset/division_duplicated.tex}
        \end{tikzpicture}
        \caption{Subgraph for merger resolution: Divisions are duplicated
            $\left(\protect\usebox{\captionF}\right)$. Dashed circles indicate nodes that are not
            part of the merger resolving process. Note that the duplicated nodes $X_2^t$ and
            $\bar{X}_2^t$ are identical and can be interchanged without alteration of the tracking
            result.}
        \label{subfig:gmm-division-duplicate}
    \end{subfigure}
    \hfill
    \begin{subfigure}[t]{0.32\textwidth}
        \centering
        \begin{tikzpicture}[minimum size=58pt,scale=0.35, every node/.style={scale=0.35, text=black, font=\LARGE}, thick]
            \input{images/gmm/gmm_hyp_graph_subset/division_inferred.tex}
        \end{tikzpicture}
        \caption{Result after merger resolving and reconstruction of cell identities: The duplicates
        of $X_2^t$ are reunited in a single node.}
        \label{subfig:gmm-division-inferred}
    \end{subfigure}
    %\end{tabular}
    %\rule{\textwidth}{0.3pt}
    \caption[Division into a merged object]{Resolution of a merger event involving a division of an
        object in the previous time step into a merger
        object~(\subref{subfig:gmm-division-into-merger}). First, the division node is
        duplicated~(\subref{subfig:gmm-division-duplicate}), then inference on the subgraph --
        including one of the duplicates -- yields a tracking solution. Finally, the duplicates of
        the division node are reunited~(\subref{subfig:gmm-division-inferred}).}
    \label{fig:gmm-division-duplication}
\end{figure}

Using this modified subgraph $G_{\text{sub}}$, a factor graph is built in the manner of the
conservation tracking with modified constraints:
\begin{enumerate}
      \item The conservation tracking marks all active detections true positive and the identities of
    merged objects are reconstructed. Therefore, each detection variable in $G_{\text{sub}}$ is
    fixed to one (Equation~\ref{eq:true-detection-constraint}).
      \item Divisions are disallowed (Equation~\ref{eq:no-division-constraint}) and division into merged
    objects is modeled by duplication of the parent node.
      \item With the cell identities reconstructed, each active transition variable corresponds to
    the transition of an object of mass one (Equation~\ref{eq:mass-one-constraint}), \ie a single cell, and
      \item Cells cannot merge or divide. Therefore, exactly one assignment variable is active in
    each outgoing (Equation~\ref{eq:cons-sub-out-a}) and incoming (Equation~\ref{eq:cons-sub-in-a})
    factor.
\end{enumerate}
In the sense of the conservation tracking factor graph, the constraints on the value range of the
random variables can be formulated as
\begin{subequations} \label{eq:main}
    \begin{alignat}{2}
    X_i^t = &1 \; \forall \; X_i^t \in \mathcal{X}_{\text{sub}}, \label{eq:true-detection-constraint}\\
    D_i^t =& 0 \; \forall \; i, t: X_i^t \in \mathcal{X}_{\text{sub}}, \label{eq:no-division-constraint}\\
    0 \le T_{ij}^t \le& 1 \; \forall \;  T_{ij}^t \in \mathcal{T}_{\text{sub}}. \label{eq:mass-one-constraint}
\end{alignat}
\end{subequations}
Furthermore, the constraints on outgoing and incoming assignments gives rise to the formulation of
new outgoing energies
\begin{subnumcases}{\label{eq:cons-sub-out} E_{\text{out}}(A_i^t, T_{ij_1}^t,\dots ,T_{ij_n}^t) =}
    \infty, & $\sum_{l\in \{j_1,\dots , j_n\}} T_{il}^t \ne A_i^t + D_i^t = 1$ \label{eq:cons-sub-out-a} \\
    0, & $\text{otherwise}$ \label{eq:cons-sub-out-e}
\end{subnumcases}
and incoming energies
\begin{subnumcases}{\label{eq:cons-sub-in} E_{\text{in}}(V_j^{t+1}, T_{i_1j}^t,\dots ,T_{i_nj}^t) =}
    \infty, & $\sum_{l\in \{j_1,\dots , j_n\}} T_{il}^t \ne V_i^{t+1} + D_i^{t+1} = 1$ \label{eq:cons-sub-in-a} \\
    0, & $\text{otherwise}$ \label{eq:cons-sub-in-e}
\end{subnumcases}
which ensure that, if there are potential incoming or outgoing assignments, exactly one of each is
active.

The remaining factors of the conservation tracking factor graph, namely,
$\psi_{\text{det}}$~(\cref{eq:fg-conservation-det}),
$\phi_{\text{tr}}$~(\cref{eq:fg-conservation-trans}) and
$\phi_{\text{div}}$~(\cref{eq:fg-conservation-div}), are left unchanged. Then, with all variables
except for the transition variables being fixed to a certain state by constraints, minimizing the energy
\begin{align} \argmax_{\mathcal{A}_{\text{sub}}, \mathcal{V}_{\text{sub}},
        \mathcal{D}_{\text{sub}}, \mathcal{T}_{\text{sub}}}P(\mathcal{A}_{\text{sub}},
    \mathcal{V}_{\text{sub}},
    \mathcal{D}_{\text{sub}}, \mathcal{T}_{\text{sub}})  &= \argmax_{\mathcal{T}_{\text{sub}}} P(\mathcal{T}_{\text{sub}})\\
    &=\argmin_{\mathcal{T}_{\text{sub}}} E(\mathcal{T}_{\text{sub}}) \\
    &=\argmin_{\mathcal{T}_{\text{sub}}}\left(-\log P\left(\mathcal{T}_{\text{sub}}\right)\right).
    \label{eq:fg-conservation-map-resolve}
\end{align}
for the MAP solution in an ILP is a function of the transition variables only. Finally, inactive
transitions are discarded from the graph
(\cref{fig:gmm-hypotheses-graph-subset-after-inference}). This concluding graph yields an optimal
solution for the assignment problem and furthermore contains the reconstructed identities of cells
that were originally lost due to undersegmentation.

With the introduction of the conservation tracking with cell identity reconstruction completed,
\cref{sec:gmm-experiments} demonstrates the performance of the method on two challenging data sets.

\begin{figure}
    \centering
    \scalebox{0.7}{
        \begin{tikzpicture}[minimum size=58pt,scale=0.45, every node/.style={scale=0.45, text=black, font=\LARGE}, thick]
            
\begin{scope}
    \node (t1) {\huge $t$};
    \node[hypotheses_detection_not_involved, below=of t1] (x11) {$X_1^t$};
    \node[hypothesesdetection, below=of x11, circle, draw] (x12) {$X_2^t$};
    \node[hypothesesdetection, below=of x12, circle, draw] (x13) {$X_3^t$};
    \node[hypotheses_detection_not_involved, below=of x13, circle, draw] (x14) {$X_4^t$};
\end{scope}


\begin{scope}
    \node[right=of t1, xshift=15mm] (t2) {\huge $t+1$};
    \node[hypotheses_detection_not_involved, below=of t2] (x21) {$X_5^{t+1}$};
    \node[hypothesesdetection, below=of x21, circle, draw] (x22) {$X_{14}^{t+1}$};
    \node[hypothesesdetection, below=of x22, circle, draw] (x23) {$X_{15}^{t+1}$};
    \node[hypothesesdetection, below=of x23, circle, draw] (x24) {$X_{7}^{t+1}$};
\end{scope}

\begin{scope}
    \node[right=of t2, xshift=15mm] (t3) {\huge $t+2$};
    \node[hypotheses_detection_not_involved, below=of t3] (x31) {$X_8^{t+2}$};
    \node[hypothesesdetection, below=of x31, circle, draw] (x32) {$X_{16}^{t+2}$};
    \node[hypothesesdetection, below=of x32, circle, draw] (x33) {$X_{17}^{t+2}$};
    \node[hypothesesdetection, below=of x33, circle, draw] (x34) {$X_{18}^{t+2}$};
\end{scope}

\begin{scope}
    \node[right=of t3, xshift=15mm] (t4) {\huge $t+3$};
    \node[hypotheses_detection_not_involved, below=of t4] (x41) {$X_{10}^{t+3}$};
    \node[hypothesesdetection, below=of x41, circle, draw] (x42) {$X_{11}^{t+3}$};
    \node[hypothesesdetection, below=of x42, circle, draw] (x43) {$X_{12}^{t+3}$};
    \node[hypothesesdetection, below=of x43, circle, draw] (x44) {$X_{13}^{t+3}$};
\end{scope}

\begin{scope}[on background layer]
    \node[rectangle, draw, color=hypothesesbackground!40, fill=hypothesesbackground!30,
    fit=(x11) (x12) (x13) (x14), inner sep=13mm] (b1) {};
    \node[rectangle, draw, color=hypothesesbackground!40, fill=hypothesesbackground!30,
    fit=(x21) (x22) (x23) (x24), inner sep=13mm] (b2) {};
    \node[rectangle, draw, color=hypothesesbackground!40, fill=hypothesesbackground!30,
    fit=(x31) (x32) (x33) (x34), inner sep=13mm] (b3) {};
    \node[rectangle, draw, color=hypothesesbackground!40, fill=hypothesesbackground!30,
    fit=(x41) (x42) (x43) (x44), inner sep=13mm] (b4) {};
\end{scope}

\path[hypotheses_transition_not_involved] (x11) edge (x21);
\path[hypothesestransition] (x12) edge (x22);
\path[hypothesestransition] (x13) edge (x23);
\path[hypotheses_transition_not_involved] (x14) edge (x24);

\path[hypotheses_transition_not_involved] (x21) edge (x31);
\path[hypothesestransition] (x22) edge (x33);
\path[hypothesestransition] (x23) edge (x32);
\path[hypothesestransition] (x24) edge (x34);

\path[hypotheses_transition_not_involved] (x31) edge (x41);
\path[hypothesestransition] (x32) edge (x42);
\path[hypothesestransition] (x33) edge (x43);
\path[hypothesestransition] (x34) edge (x44);


%%% Local Variables: 
%%% mode: latex
%%% TeX-master: "../../../main"
%%% End: 

        \end{tikzpicture}
    }
    %\rule{\textwidth}{0.3pt}
    \caption{Subset of hypotheses graph final result after inference }
    \label{fig:gmm-hypotheses-graph-subset-after-inference}
\end{figure}



%%% Local Variables: 
%%% mode: latex
%%% TeX-master: "../../../main"
%%% End: 
