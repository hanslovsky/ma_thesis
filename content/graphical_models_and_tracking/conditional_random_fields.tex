\subsection{Conditional Random Fields}
\label{subsec:gm-crf}

Conditional random fields (CRFs) are useful for modeling conditional distributions
$P(\mathcal{Y}|\mathcal{X})$ over random output variables $(\mathcal{Y}$) conditioned on the random
input variables $\mathcal{X}$~\citep[142]{koller_09_probabilistic}. In order to
formally define CRFs, we first introduce \emph{cliques}, \emph{Markov networks} and the \emph{Markov
    property}.

\begin{mydef}[{\citealp[23]{barber_12_bayesian}}]
    \label{def:clique}
    In an undirected graph $G = (\mathcal{V}, \mathcal{E})$, a \emph{clique} is a fully connected
    subset of the vertices $\mathcal{V}_c\subseteq\mathcal{V}$, \ie there is a pairwise edge between
    all members of $\mathcal{V}_c$, or more formally: $\exists (V_i \textbf{ --- } V_j) \in
    \mathcal{E}\; \forall \; V_i \ne V_j \in \mathcal{V}_c$. A clique is called \emph{maximal
        clique}, if, for all vertices that are not in the clique $\mathcal{V} \setminus
    \mathcal{V}_c$, the union $\mathcal{V}_c^{\prime} = \mathcal{V}_c \cup \{V_a : V_a \in
    \mathcal{V} \setminus \mathcal{V}_c\}$ of the clique with an arbitrary vertex is not a
    clique. The largest maximal clique in a graph is called the \emph{maximum clique}.
\end{mydef}

With \cref{def:clique}, we now can define a \emph{Markov network} and its graph representation.

\begin{mydef}[{\citealp[59]{barber_12_bayesian}}]
    \label{def:markov-network}
    A \emph{Markov network} over random variables $\mathcal{X}$ is a distribution that factorizes
    into non-negative potentials (functions) $\phi_i(\mathcal{X}_i)$ on subsets of the random
    variables $\mathcal{X}_i\subseteq\mathcal{X}$:
    \begin{align}
        \label{eq:markov-network}
        P(\mathcal{X}) = \frac{1}{Z}\prod_{i=1}^N\phi_i(\mathcal{X}_i)
    \end{align}
    The constant $Z$ ensures normalization. In an undirected graph representation $G$, the subsets
    of variables $\mathcal{X}_i\subseteq\mathcal{X}$ correspond to the maximal cliques in $G$.
\end{mydef}
The Markov property can be defined in terms of a Markov network.
\begin{mydef}[{\citealp[61]{barber_12_bayesian}; \citealp[16]{hammersley_71_markov}}]
    \label{def:markov-property}
    Let ($\mathcal{A}\subseteq\mathcal{X}$, $\mathcal{B}\subseteq\mathcal{X}$,
    $\mathcal{S}\subseteq\mathcal{X}$), represented by vertices ($\mathcal{V}_{\mathcal{A}}$,
    $\mathcal{V}_{\mathcal{B}}$, $\mathcal{V}_{\mathcal{S}}$), be a set of disjoint subsets of
    variables of a Markov network. Then, the \emph{Markov property} states that $\mathcal{A}
    \independent \mathcal{B} | \mathcal{S}$, if there is no path from a member of
    $\mathcal{V}_{\mathcal{A}}$ to a member of $\mathcal{V}_{\mathcal{B}}$ or if every path of any
    member of $\mathcal{V}_{\mathcal{A}}$ to any member of $\mathcal{V}_{\mathcal{B}}$ passes
    through $\mathcal{V}_{\mathcal{S}}$).
\end{mydef}

In particular, % for positive potentials $\phi_k > 0$,
a random variable in a Markov network conditioned on its neighbors is independent of the remaining
variable:
\begin{align}
    \label{eq:local-markov-property}
    P(X|\mathcal{X}\setminus X) = P(X|\neighbor(X)) \\
    \neighbor(X) = \{X^{\prime} \in \mathcal{X} : \exists (V^{\prime} \textbf{ --- } V) \in \mathcal{E}\}
\end{align}
With the knowledge about Markov networks, \emph{conditional random fields} can be defined in terms
of undirected graphical models.

\begin{mydef}[{{\citealp[234]{barber_12_bayesian}; \citealp[3]{lafferty_01_conditional}}}]
    \label{def:crf}
    Let $\mathcal{X}$ be a set of observable (input) random variables and $\mathcal{Y}$ be a set of
    label (output) random variables. Then a \emph{conditional random field} is a conditional
    distribution
    \begin{align}
        \label{eq:crf}
        P(\mathcal{Y}|\mathcal{X}) = \frac{1}{Z(X)}\prod_k\phi(\mathcal{Y}_k, \mathcal{X}_k)
    \end{align}
    over the output $\mathcal{Y}$ conditioned on the input $\mathcal{X}$ that factorizes into
    non-negative potentials $\phi_k$. A CRF can be seen as a Markov network on the output variables
    $\mathcal{Y}$ conditioned on the input $\mathcal{X}$ with a graphical representation
    $G=(\mathcal{V}, \mathcal{E})$. More precisely, the set of vertices
    $\mathcal{V}=\{\mathcal{V}_{\mathcal{Y}}, \mathcal{V}_{\mathcal{X}}\}$ comprises the subsets
    that form vertices representing output variables $\mathcal{V}_{\mathcal{Y}}$ and input variables
    $\mathcal{V}_{\mathcal{X}}$. Therefore, the Markov property holds for $\mathcal{Y}$ conditioned
    on $\mathcal{X}$ with $G_{\mathcal{Y}}=(\mathcal{V}_{\mathcal{Y}}, \mathcal{E}_{\mathcal{Y}})$.
    % Let $G = (\mathcal{V}, \mathcal{E})$ be an undirected graph, whose vertices $V\in\mathcal{V}$
    % index a set of random variables $Y_V \in \mathcal{Y}$ over labels, and a set of random variables
    % $\mathcal{X}$ over observations. Furthermore, the Markov property holds for $\mathcal{Y}$
    % conditioned on $\mathcal{X}$. Then $(X,Y)$ is a \emph{conditional random field}.
    % \begin{align}
    %     \label{eq:gm-crf}
    %     p(\mathcal{Y}|\mathcal{X}) = \frac{1}{Z(\mathcal{X})}\prod_k\phi_k(\mathcal{Y}_k, \mathcal{X}_k)
    % \end{align}
    % over observed variables $\mathcal{X}$ and random variables $\mathcal{Y}$ consisting of $k$
    % factors $phi_k$ that are non-negative functions of subsets $\mathcal{X}_k \subseteq \mathcal{X}$
    % and $\mathcal{Y}_k \subseteq \mathcal{Y}$.
\end{mydef}
Normalization is ensured in \cref{eq:crf} by division by the partition function
\begin{align}
    \label{eq:gm-crf-partition}
    Z(\mathcal{X}) = \sum_{\mathcal{Y}}\prod_k\phi_k(\mathcal{Y}_k, \mathcal{X}_k),
\end{align}
which is calculated by summation over all possible states of the output variables $\mathcal{Y}$.

\begin{figure}
    \centering
    \begin{subfigure}{0.48\textwidth}
        \centering
        \begin{tikzpicture}[thick, on grid, every node/.style={font=\small, scale=1.5}, baseline=(v.south)]
            \begin{scope}
    \node[RVnode] (v) {$y_1$};
    \node[RVnode, above=of v.north] (v1) {$y_2$};
    \node[RVnode, xshift=-40, fill=black!30] at ($(v)!0!(v1)$) (h) {$x_1$};
    \node[RVnode, xshift=-40, fill=black!30] at ($(v)!1!(v1)$) (h1) {$x_2$};
    \path (h) edge (v);
    \path (h1) edge (v1);
    \path (v) edge (v1);

\end{scope}


%%% DIRTY HACK TO ALLOW FOR CAPTIONS AT SAME POSITION!
\begin{scope}[on background layer]
    \node[Factor, color=black!0, label={[font=\tiny, xshift=7, yshift=-7, color=black!0]below:$\psi_1(x_1,y_1)$}] (f1) at
    ($(v)!0.5!(h)$) {};
    \node[Factor, color=black!0, label={[font=\tiny, xshift=7, yshift=7, color=black!0]above:$\psi_1(x_1,y_1)$}] (f2) at
    ($(v1)!0.5!(h1)$) {};
    
\end{scope}

%%% Local Variables: 
%%% mode: latex
%%% TeX-master: "../../main"
%%% End: 

        \end{tikzpicture}
        %\rule{\textwidth}{0.3pt}
        \caption{CRF}
        \label{fig:gm-crf-example}
    \end{subfigure}
    ~
    \begin{subfigure}{0.48\textwidth}
        \centering
        \begin{tikzpicture}[thick, on grid, every node/.style={font=\small, scale=1.5}, baseline=(v.south)]
            \begin{scope}
    \node[RVnode] (v) {$y_1$};
    \node[RVnode, above=of v.north] (v1) {$y_2$};
    \node[RVnode, xshift=-40, fill=black!30] at ($(v)!0!(v1)$) (h) {$x_1$};
    \node[RVnode, xshift=-40, fill=black!30] at ($(v)!1!(v1)$) (h1) {$x_2$};
    \node[Factor, label={[font=\tiny, xshift=7, yshift=-7]below:$\psi_1(x_1,y_1)$}] (f1) at ($(v)!0.5!(h)$) {};
    \node[Factor, label={[font=\tiny, xshift=7, yshift=7]above:$\psi_2(x_2,y_2)$}] (f2) at ($(v1)!0.5!(h1)$) {};
    \node[Factor, label={[font=\tiny]right:$\psi_3(x_1, x_2)$}] at ($(v)!0.5!(v1)$) (f3) {};
    \factoredge {v, h} {f1} {};
    \factoredge {v1, h1} {f2} {};
    \factoredge {v1, v} {f3} {};

\end{scope}

%%% Local Variables: 
%%% mode: latex
%%% TeX-master: "../../main"
%%% End: 

        \end{tikzpicture}
        %\rule{\textwidth}{0.3pt}
        \caption{Factor graph representation for the CRF}
        \label{fig:gm-crf-example-factor}
    \end{subfigure}
    \caption[Conditional random field example]{Conditional random field example
        (\subref{fig:gm-crf-example}) with input (gray) and output variables (beige), following the
        distribution $p(y_1, y_2|x_1, x_2) = \frac{1}{Z}\phi_1(x_1, y_1)\phi_2(x_2,
        y_2)\phi(y_1,y_2)$. Its factor graph (\cref{subsec:factor-graphs}) representation is
        shown in (\subref{fig:gm-crf-example-factor}).}
    \label{fig:gm-crf}
\end{figure}


%%% Local Variables: 
%%% mode: latex
%%% TeX-master: "../../main"
%%% End: 
