\subsection{Bayesian Networks}
\label{subsec:gm-bayesian-net}

Bayesian Networks are graphical models that are used for modeling causal independencies. The
structure of the graph defines the conditional dependencies of the underlying distribution.

\begin{mydef}[{\citealp[37]{barber_12_bayesian}}]
    A \emph{Bayesian network}, also \emph{belief network} or \emph{directed acyclic graphical model}
    (DAG), is a graphical model that describes the distribution
    \begin{align}
        \label{eq:gm-bayesian-net}
        P(\mathcal{X}) = \prod_{X \in \mathcal{X}}P\left(X=x|\pa(X)\right).
    \end{align}
    over a set of random variables $\mathcal{X}$. In the graph representation
    \begin{align}
        G = (\mathcal{V}, \mathcal{A}),
    \end{align}
    each vertex $V \in \mathcal{V}$ represents the distribution over a random variable $X_V \in
    \mathcal{X}$ conditioned on the set of parental variables of $X_V$ as indicated by arrows from
    each parent to child $V$.
    % a random variable $X_V \in \mathcal{X}$. Furthermore,
    % the set of arcs $\mathcal{A}$ is constructed in a way, such that the parents of each node
    % $X_{V_i}\in\mathcal{X}$ are the $M$ vertices $\{V_1,\hdots,V_M\}$ that represent the random
    % variables that $X_{V_i}$ is conditioned on.
\end{mydef}

In that context, $\pa(X_V)$ is the set of parental variables for the random variable
$X_V\in\mathcal{X}$. In terms of a directed acyclic graph
$G=\left(\mathcal{V},\mathcal{A}\right)$, this is the set of all random variables represented by
nodes with an arc pointing to $V$
\begin{align}
    \label{gm-bayesian-parental}
    \pa(X_V) = \left\{X_{V_i} \in \mathcal{X} : (V_i \to V) \in \mathcal{A} \right\}.
\end{align}
This structure is visualized by a toy example in \cref{fig:gm-bayesian-net}.  Furthermore, it can be
exploited to determine conditional independencies of the random variables. In this regard,
\citet{verma_88_causal} introduced the algorithm called \emph{d-separation} for the conclusion of
conditional dependencies from a DAG. \citet[43]{barber_12_bayesian} provides a more compact
description of the method.
\begin{figure}
    \centering
    \begin{subfigure}[t]{0.48\textwidth}
        \centering
        \begin{tikzpicture}[thick, on grid, every node/.style={font=\small, scale=1.5}, baseline=(v.south)]
            \begin{scope}
    \node[RVnode] (v) {$X_1$};
    \node[RVnode, above=of v.north] (v1) {$X_2$};
    \node[RVnode, xshift=-40] at ($(v)!0.5!(v1)$) (h) {$X_3$};
    \path (h) edge[connect] (v);
    \path (h) edge[connect] (v1);
\end{scope}

%%% Local Variables: 
%%% mode: latex
%%% TeX-master: "../../main"
%%% End: 

        \end{tikzpicture}
        %\rule{\textwidth}{0.3pt}
        \caption{Bayesian network: Both $X_1$ and $X_2$ have $X_3$ as a parent.}
        \label{subfig:gm-bayesian-net-example}
    \end{subfigure}
    ~
    \begin{subfigure}[t]{0.48\textwidth}
        \centering
        \begin{tikzpicture}[thick, on grid, every node/.style={font=\small, scale=1.5}, baseline=(v.south)]
            \begin{scope}
    \node[RVnode] (v) {$X_1$};
    \node[RVnode, above=of v.north] (v1) {$X_2$};
    \node[RVnode, xshift=-40] at ($(v)!0.5!(v1)$) (h) {$X_3$};
    \node[Factor, label={[font=\tiny]below left:$\phi_1(x_1,x_3)$}] (f1) at ($(v)!0.5!(h)$) {};
    \node[Factor, label={[font=\tiny]above left:$\phi_2(x_2,x_3)$}] (f2) at ($(v1)!0.5!(h)$) {};
    \node[Factor, label={[font=\tiny]left:$\phi_3(x_1)$}, left=of h.west] (f3) {};
    \factoredge {v, h} {f1} {};
    \factoredge {v1, h} {f2} {};
    \factoredge {h} {f3} {};
\end{scope}

%%% Local Variables: 
%%% mode: latex
%%% TeX-master: "../../main"
%%% End: 

        \end{tikzpicture}
        %\rule{\textwidth}{0.3pt}
        \caption{Factor graph (\cref{subsec:factor-graphs}) representation of
            \cref{subfig:gm-bayesian-net-example}.}
        \label{subfig:gm-bayesian-net-fg}
    \end{subfigure}
    \caption[A simple Bayesian network]{A simple Bayesian network
        (\subref{subfig:gm-bayesian-net-example}) representing the distribution $P(x_1,x_2,x_3) =
        P(x_1|x_3)P(x_2|x_3)P(x_3)$ and a factor graph representation (\subref{subfig:gm-bayesian-net-fg}) $P(x_1, x_2, x_3) =
        \frac{1}{Z}\phi_1(x_1,x_3)\phi_2(x_2,x_3)\phi_3(x_3)$.}
    \label{fig:gm-bayesian-net}
\end{figure}



%%% Local Variables: 
%%% mode: latex
%%% TeX-master: "../../main"
%%% End: 
