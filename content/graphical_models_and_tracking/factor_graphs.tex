\subsection{Factor Graphs}
\label{subsec:factor-graphs}
Finally, to conclude the digression on graphical models, this section introduces factor graphs.

\begin{mydef}[{\citealp{kschischang_01_factor,frey_98_factor}}]
    \label{def:factor-graph}
    A \emph{factor graph} is an undirected bipartite graph $G=(\{\mathcal{U},\mathcal{V}\},
    \mathcal{E})$, \ie the set of vertices is formed by two disjoint sets $\mathcal{U}$ and
    $\mathcal{V}$ with $\mathcal{U} \cap \mathcal{V} = \emptyset$, such that there is no edge
    between two vertices from the same set $\mathcal{U}$ or $\mathcal{V}$:
\begin{align}
    \label{eq:bipartite}
    \nexists (U_1 \undirected U_2) \in \mathcal{E} \; \forall \; U_1, U_2 \in
    \mathcal{U} \wedge \nexists (V_1 \undirected V_2) \in \mathcal{E} \; \forall \; V_1, V_2 \in
    \mathcal{V}
\end{align}
It represents a probability distribution over random variables $\mathcal{X} = \{X_V : V \in \mathcal{V}\}$
\begin{align}
    \label{eq:fg-distribution}
    P(\mathcal{X}) = \frac{1}{Z}\prod_{U \in \mathcal{U}}\psi_U(\mathcal{X}_U),
\end{align}
with the vertices $\mathcal{U}$ and $\mathcal{V}$ indexing the non-negative \emph{potentials} or
\emph{factors} $\psi_\mathcal{U} = \{\psi_U : U \in \mathcal{U}\}$ and the random variables
respectively. Here, $\psi_U$ is a function of the set of random variables
$\mathcal{X}_U$. Furthermore, the set of edges $\mathcal{E}$ is defined such that the vertex $U$
representing a potential $\psi_U$ is connected to the vertices $\mathcal{V}_{\mathcal{X_U}}$ of all
of the potential's arguments $\mathcal{X}_U$:
\begin{align}
    \label{eq:fg-edges}
    \mathcal{E}_U &= \left\{(U \undirected V ) : V \in {V}_{\mathcal{X_U}}\right\}\\
    \mathcal{E} &= \left\{\mathcal{E}_U : U \in \mathcal{U}\right\}
\end{align}
\end{mydef}

With \cref{def:factor-graph}, it becomes clear, that the structure of a factor graph reflects the
factorization of a distribution into potentials. In general, the following conventions are agreed
upon for the illustrations of factor graphs:
\begin{enumerate}
      \item Random variable nodes $\mathcal{V}$ are typically denoted by circles.
      \item Factor nodes $\mathcal{U}$ are typically denoted by filled squares.
\end{enumerate}
In order to illustrate these conventions, \cref{fig:fg-simple} shows an example of a simple factor
graph with the probability distribution
\begin{align}
    \label{eq:fg-simple}
    P(x_1,x_2,x_3,x_4,x_5) = \frac{1}{Z}\psi_1(x_1,x_2,x_3)\psi_2(x_1,x_4)\psi_3(x_5).
\end{align}
Moreover, factor graphs implement dependencies and interactions of random variables in a very
intuitive manner in terms of factors. In addition, a range of inference algorithms is applicable to
factor graphs, \eg (loopy) \emph{belief propagation}~(\citealp{pearl_82_reverend};
\citealp[Chapter~5.1.3,Chapter~28.7]{barber_12_bayesian}), \emph{belief revision}
(\citealp{darwiche_96_logic}, \citealp[Chapter~5.2.1]{barber_12_bayesian}), and both
Bayesian networks and Markov networks can be transformed into extended factor graphs without loss of
expressiveness (\citealp[Section~2]{frey_03_extending}). In the case of Markov networks, factors
represent cliques in the original graph. However, multiple factors can be merged into a single
factor and then the direct mapping from cliques to factors does not hold anymore. This makes the
factor graph a powerful tool both for modeling and inference.

Furthermore, the probability distribution of a factor graph can be interpreted as a Gibbs
distribution
\begin{align}
    \label{fg-gibbs}
    P(\mathcal{X}) = \frac{1}{Z}e^{-E(\mathcal{X})}
\end{align}
with
\begin{align}
    E(\mathcal{X}) = -\log\left(\prod_{U \in \mathcal{U}}\psi_U(\mathcal{X}_U)\right) = -\sum_{U \in \mathcal{U}}\log\left(\psi_U(\mathcal{X}_U)\right).
\end{align}
Then, the MAP solution
\begin{align}
    \label{fg-map}
    \argmax_{\mathcal{X}}P(\mathcal{X}) = \argmin_{\mathcal{X}}E(\mathcal{X})
\end{align}
can be calculated using \emph{linear programming}~(LP, \cref{cha:app-ilp}) or \emph{integer linear
    programming}~(ILP, \cref{cha:app-ilp}) for discrete random
variables. In the following, we make use of this transformation to exploit the advantage that,
instead of using zero probability (infinite energy) to forbid certain configurations, we can
directly exclude these configurations from the feasible regions by the means of linear constraints
(also hard constraints).
% can be calculated using integer linear programming (ILP,
% \citet{vanderbei_07_linear}{Chapter~22}). Throughout this thesis, all presented
% tracking approaches  





\begin{figure}
    \centering
    \begin{tikzpicture}[thick, on grid, every node/.style={font=\small, scale=1.5}]
        {% \scalefont{0.5}
    \node[RVnode] (rv1) {$x_1$};
    \node[RVnode, above right=of rv1.north east, yshift=10] (rv2) {$x_2$};
    \node[RVnode, above left=of rv1.north west, yshift=10] (rv3) {$x_3$};
    \node[RVnode, below=of rv1.south] (rv4) {$x_4$};
    \node[RVnode, right=of rv2.east, xshift=20] (rv5) {$x_5$};
    \node[Factor, label={[font=\tiny]left:$\psi_2(x_1,x_4)$}] (f1) at ($(rv1)!0.5!(rv4)$) {};
    \node[Factor, label={[font=\tiny]below right:$\psi_1(x_1,x_2,x_3)$}] (f2) at ($(rv2)!0.5!(rv3)!0.3!(rv1)$) {};
    \node[Factor, below=of rv5.south, label={[font=\tiny]right:$\psi_3(x_5)$}] (f3) {};
    \factoredge {rv1, rv4} {f1} {};
    \factoredge {rv1, rv2, rv3} {f2} {};
    \factoredge {rv5} {f3} {};
    % \path[draw, very thick] (rv1) edge (f1);
    % \path[draw, very thick] (rv1) edge (f2);
    % \path[draw, very thick] (rv2) edge (f2);
    % \path[draw, very thick] (rv3) edge (f2);
    % \path[draw, very thick] (rv4) edge (f1);
    % \path[draw, very thick] (rv5) edge (f3);
}

%%% Local Variables: 
%%% mode: latex
%%% TeX-master: "../../../main"
%%% End: 

    \end{tikzpicture}
    %\rule{\textwidth}{0.3pt}
    \caption[Example of a simple factor graph]{Example of a simple factor graph containing five
        random variables (circles with beige fill) and three factors (black squares).}
    \label{fig:fg-simple}
\end{figure}




%%% Local Variables: 
%%% mode: latex
%%% TeX-master: "../../main"
%%% End: 
