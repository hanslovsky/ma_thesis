\section{Graphical Models}
\label{sec:gm-graphical-models}

A graphical representation of probability distributions is helpful for modeling
independence/dependence relationships. Graphical models impose structure that reflects those
relations. However, the capabilities of graphical models is limited and any use case requires the
choice of an appropriate kind of graphical model \citep[57]{barber_12_bayesian}. Of
the many types of graphical models we present Bayesian networks (\cref{subsec:gm-bayesian-net}),
conditional random fields (\cref{subsec:gm-crf}) and factor graphs
(\cref{subsec:factor-graphs}). Both the Bayesian networks and the conditional random fields can be
transformed into a factor graph representation, which is commonly used for inference. Moreover, as
shown in \cref{subsec:fg-conservation} and \cref{cha:joint}, a factor graph can also be utilized for
modeling.

\subsection{Bayesian Networks}
\label{subsec:gm-bayesian-net}

Bayesian Networks are graphical models that are used for modeling causal independencies. The
structure of the graph defines the conditional dependencies of the underlying distribution.

\begin{mydef}[{\citealp[37]{barber_12_bayesian}}]
    A \emph{Bayesian network}, also \emph{belief network} or \emph{directed acyclic graphical model}
    (DAG), is a graphical model that describes the distribution
    \begin{align}
        \label{eq:gm-bayesian-net}
        P(\mathcal{X}) = \prod_{X \in \mathcal{X}}P\left(X=x|\pa(X)\right).
    \end{align}
    over a set of random variables $\mathcal{X}$. In the graph representation
    \begin{align}
        G = (\mathcal{V}, \mathcal{A}),
    \end{align}
    each vertex $V \in \mathcal{V}$ represents the distribution over a random variable $X_V \in
    \mathcal{X}$ conditioned on the set of parental variables of $X_V$ as indicated by arrows from
    each parent to child $V$.
    % a random variable $X_V \in \mathcal{X}$. Furthermore,
    % the set of arcs $\mathcal{A}$ is constructed in a way, such that the parents of each node
    % $X_{V_i}\in\mathcal{X}$ are the $M$ vertices $\{V_1,\hdots,V_M\}$ that represent the random
    % variables that $X_{V_i}$ is conditioned on.
\end{mydef}

In that context, $\pa(X_V)$ is the set of parental variables for the random variable
$X_V\in\mathcal{X}$. In terms of a directed acyclic graph
$G=\left(\mathcal{V},\mathcal{A}\right)$, this is the set of all random variables represented by
nodes with an arc pointing to $V$
\begin{align}
    \label{gm-bayesian-parental}
    \pa(X_V) = \left\{X_{V_i} \in \mathcal{X} : (V_i \to V) \in \mathcal{A} \right\}.
\end{align}
This structure is visualized by a toy example in \cref{fig:gm-bayesian-net}.  Furthermore, it can be
exploited to determine conditional independencies of the random variables. In this regard,
\citet{verma_88_causal} introduced the algorithm called \emph{d-separation} for the conclusion of
conditional dependencies from a DAG. \citet[43]{barber_12_bayesian} provides a more compact
description of the method.
\begin{figure}
    \centering
    \begin{subfigure}[t]{0.48\textwidth}
        \centering
        \begin{tikzpicture}[thick, on grid, every node/.style={font=\small, scale=1.5}, baseline=(v.south)]
            \begin{scope}
    \node[RVnode] (v) {$X_1$};
    \node[RVnode, above=of v.north] (v1) {$X_2$};
    \node[RVnode, xshift=-40] at ($(v)!0.5!(v1)$) (h) {$X_3$};
    \path (h) edge[connect] (v);
    \path (h) edge[connect] (v1);
\end{scope}

%%% Local Variables: 
%%% mode: latex
%%% TeX-master: "../../main"
%%% End: 

        \end{tikzpicture}
        %\rule{\textwidth}{0.3pt}
        \caption{Bayesian network: Both $X_1$ and $X_2$ have $X_3$ as a parent.}
        \label{subfig:gm-bayesian-net-example}
    \end{subfigure}
    ~
    \begin{subfigure}[t]{0.48\textwidth}
        \centering
        \begin{tikzpicture}[thick, on grid, every node/.style={font=\small, scale=1.5}, baseline=(v.south)]
            \begin{scope}
    \node[RVnode] (v) {$X_1$};
    \node[RVnode, above=of v.north] (v1) {$X_2$};
    \node[RVnode, xshift=-40] at ($(v)!0.5!(v1)$) (h) {$X_3$};
    \node[Factor, label={[font=\tiny]below left:$\phi_1(x_1,x_3)$}] (f1) at ($(v)!0.5!(h)$) {};
    \node[Factor, label={[font=\tiny]above left:$\phi_2(x_2,x_3)$}] (f2) at ($(v1)!0.5!(h)$) {};
    \node[Factor, label={[font=\tiny]left:$\phi_3(x_1)$}, left=of h.west] (f3) {};
    \factoredge {v, h} {f1} {};
    \factoredge {v1, h} {f2} {};
    \factoredge {h} {f3} {};
\end{scope}

%%% Local Variables: 
%%% mode: latex
%%% TeX-master: "../../main"
%%% End: 

        \end{tikzpicture}
        %\rule{\textwidth}{0.3pt}
        \caption{Factor graph (\cref{subsec:factor-graphs}) representation of
            \cref{subfig:gm-bayesian-net-example}.}
        \label{subfig:gm-bayesian-net-fg}
    \end{subfigure}
    \caption[A simple Bayesian network]{A simple Bayesian network
        (\subref{subfig:gm-bayesian-net-example}) representing the distribution $P(x_1,x_2,x_3) =
        P(x_1|x_3)P(x_2|x_3)P(x_3)$ and a factor graph representation (\subref{subfig:gm-bayesian-net-fg}) $P(x_1, x_2, x_3) =
        \frac{1}{Z}\phi_1(x_1,x_3)\phi_2(x_2,x_3)\phi_3(x_3)$.}
    \label{fig:gm-bayesian-net}
\end{figure}



%%% Local Variables: 
%%% mode: latex
%%% TeX-master: "../../main"
%%% End: 

\subsection{Conditional Random Fields}
\label{subsec:gm-crf}

Conditional random fields (CRFs) are useful for modeling conditional distributions
$P(\mathcal{Y}|\mathcal{X})$ over random output variables $(\mathcal{Y}$) conditioned on the random
input variables $\mathcal{X}$~\citep[142]{koller_09_probabilistic}. In order to
formally define CRFs, we first introduce \emph{cliques}, \emph{Markov networks} and the \emph{Markov
    property}.

\begin{mydef}[{\citealp[23]{barber_12_bayesian}}]
    \label{def:clique}
    In an undirected graph $G = (\mathcal{V}, \mathcal{E})$, a \emph{clique} is a fully connected
    subset of the vertices $\mathcal{V}_c\subseteq\mathcal{V}$, \ie there is a pairwise edge between
    all members of $\mathcal{V}_c$, or more formally: $\exists (V_i \textbf{ --- } V_j) \in
    \mathcal{E}\; \forall \; V_i \ne V_j \in \mathcal{V}_c$. A clique is called \emph{maximal
        clique}, if, for all vertices that are not in the clique $\mathcal{V} \setminus
    \mathcal{V}_c$, the union $\mathcal{V}_c^{\prime} = \mathcal{V}_c \cup \{V_a : V_a \in
    \mathcal{V} \setminus \mathcal{V}_c\}$ of the clique with an arbitrary vertex is not a
    clique. The largest maximal clique in a graph is called the \emph{maximum clique}.
\end{mydef}

With \cref{def:clique}, we now can define a \emph{Markov network} and its graph representation.

\begin{mydef}[{\citealp[59]{barber_12_bayesian}}]
    \label{def:markov-network}
    A \emph{Markov network} over random variables $\mathcal{X}$ is a distribution that factorizes
    into non-negative potentials (functions) $\phi_i(\mathcal{X}_i)$ on subsets of the random
    variables $\mathcal{X}_i\subseteq\mathcal{X}$:
    \begin{align}
        \label{eq:markov-network}
        P(\mathcal{X}) = \frac{1}{Z}\prod_{i=1}^N\phi_i(\mathcal{X}_i)
    \end{align}
    The constant $Z$ ensures normalization. In an undirected graph representation $G$, the subsets
    of variables $\mathcal{X}_i\subseteq\mathcal{X}$ correspond to the maximal cliques in $G$.
\end{mydef}
The Markov property can be defined in terms of a Markov network.
\begin{mydef}[{\citealp[61]{barber_12_bayesian}; \citealp[16]{hammersley_71_markov}}]
    \label{def:markov-property}
    Let ($\mathcal{A}\subseteq\mathcal{X}$, $\mathcal{B}\subseteq\mathcal{X}$,
    $\mathcal{S}\subseteq\mathcal{X}$), represented by vertices ($\mathcal{V}_{\mathcal{A}}$,
    $\mathcal{V}_{\mathcal{B}}$, $\mathcal{V}_{\mathcal{S}}$), be a set of disjoint subsets of
    variables of a Markov network. Then, the \emph{Markov property} states that $\mathcal{A}
    \independent \mathcal{B} | \mathcal{S}$, if there is no path from a member of
    $\mathcal{V}_{\mathcal{A}}$ to a member of $\mathcal{V}_{\mathcal{B}}$ or if every path of any
    member of $\mathcal{V}_{\mathcal{A}}$ to any member of $\mathcal{V}_{\mathcal{B}}$ passes
    through $\mathcal{V}_{\mathcal{S}}$).
\end{mydef}

In particular, % for positive potentials $\phi_k > 0$,
a random variable in a Markov network conditioned on its neighbors is independent of the remaining
variable:
\begin{align}
    \label{eq:local-markov-property}
    P(X|\mathcal{X}\setminus X) = P(X|\neighbor(X)) \\
    \neighbor(X) = \{X^{\prime} \in \mathcal{X} : \exists (V^{\prime} \textbf{ --- } V) \in \mathcal{E}\}
\end{align}
With the knowledge about Markov networks, \emph{conditional random fields} can be defined in terms
of undirected graphical models.

\begin{mydef}[{{\citealp[234]{barber_12_bayesian}; \citealp[3]{lafferty_01_conditional}}}]
    \label{def:crf}
    Let $\mathcal{X}$ be a set of observable (input) random variables and $\mathcal{Y}$ be a set of
    label (output) random variables. Then a \emph{conditional random field} is a conditional
    distribution
    \begin{align}
        \label{eq:crf}
        P(\mathcal{Y}|\mathcal{X}) = \frac{1}{Z(X)}\prod_k\phi(\mathcal{Y}_k, \mathcal{X}_k)
    \end{align}
    over the output $\mathcal{Y}$ conditioned on the input $\mathcal{X}$ that factorizes into
    non-negative potentials $\phi_k$. A CRF can be seen as a Markov network on the output variables
    $\mathcal{Y}$ conditioned on the input $\mathcal{X}$ with a graphical representation
    $G=(\mathcal{V}, \mathcal{E})$. More precisely, the set of vertices
    $\mathcal{V}=\{\mathcal{V}_{\mathcal{Y}}, \mathcal{V}_{\mathcal{X}}\}$ comprises the subsets
    that form vertices representing output variables $\mathcal{V}_{\mathcal{Y}}$ and input variables
    $\mathcal{V}_{\mathcal{X}}$. Therefore, the Markov property holds for $\mathcal{Y}$ conditioned
    on $\mathcal{X}$ with $G_{\mathcal{Y}}=(\mathcal{V}_{\mathcal{Y}}, \mathcal{E}_{\mathcal{Y}})$.
    % Let $G = (\mathcal{V}, \mathcal{E})$ be an undirected graph, whose vertices $V\in\mathcal{V}$
    % index a set of random variables $Y_V \in \mathcal{Y}$ over labels, and a set of random variables
    % $\mathcal{X}$ over observations. Furthermore, the Markov property holds for $\mathcal{Y}$
    % conditioned on $\mathcal{X}$. Then $(X,Y)$ is a \emph{conditional random field}.
    % \begin{align}
    %     \label{eq:gm-crf}
    %     p(\mathcal{Y}|\mathcal{X}) = \frac{1}{Z(\mathcal{X})}\prod_k\phi_k(\mathcal{Y}_k, \mathcal{X}_k)
    % \end{align}
    % over observed variables $\mathcal{X}$ and random variables $\mathcal{Y}$ consisting of $k$
    % factors $phi_k$ that are non-negative functions of subsets $\mathcal{X}_k \subseteq \mathcal{X}$
    % and $\mathcal{Y}_k \subseteq \mathcal{Y}$.
\end{mydef}
Normalization is ensured in \cref{eq:crf} by division by the partition function
\begin{align}
    \label{eq:gm-crf-partition}
    Z(\mathcal{X}) = \sum_{\mathcal{Y}}\prod_k\phi_k(\mathcal{Y}_k, \mathcal{X}_k),
\end{align}
which is calculated by summation over all possible states of the output variables $\mathcal{Y}$.

\begin{figure}
    \centering
    \begin{subfigure}{0.48\textwidth}
        \centering
        \begin{tikzpicture}[thick, on grid, every node/.style={font=\small, scale=1.5}, baseline=(v.south)]
            \begin{scope}
    \node[RVnode] (v) {$y_1$};
    \node[RVnode, above=of v.north] (v1) {$y_2$};
    \node[RVnode, xshift=-40, fill=black!30] at ($(v)!0!(v1)$) (h) {$x_1$};
    \node[RVnode, xshift=-40, fill=black!30] at ($(v)!1!(v1)$) (h1) {$x_2$};
    \path (h) edge (v);
    \path (h1) edge (v1);
    \path (v) edge (v1);

\end{scope}


%%% DIRTY HACK TO ALLOW FOR CAPTIONS AT SAME POSITION!
\begin{scope}[on background layer]
    \node[Factor, color=black!0, label={[font=\tiny, xshift=7, yshift=-7, color=black!0]below:$\psi_1(x_1,y_1)$}] (f1) at
    ($(v)!0.5!(h)$) {};
    \node[Factor, color=black!0, label={[font=\tiny, xshift=7, yshift=7, color=black!0]above:$\psi_1(x_1,y_1)$}] (f2) at
    ($(v1)!0.5!(h1)$) {};
    
\end{scope}

%%% Local Variables: 
%%% mode: latex
%%% TeX-master: "../../main"
%%% End: 

        \end{tikzpicture}
        %\rule{\textwidth}{0.3pt}
        \caption{CRF}
        \label{fig:gm-crf-example}
    \end{subfigure}
    ~
    \begin{subfigure}{0.48\textwidth}
        \centering
        \begin{tikzpicture}[thick, on grid, every node/.style={font=\small, scale=1.5}, baseline=(v.south)]
            \begin{scope}
    \node[RVnode] (v) {$y_1$};
    \node[RVnode, above=of v.north] (v1) {$y_2$};
    \node[RVnode, xshift=-40, fill=black!30] at ($(v)!0!(v1)$) (h) {$x_1$};
    \node[RVnode, xshift=-40, fill=black!30] at ($(v)!1!(v1)$) (h1) {$x_2$};
    \node[Factor, label={[font=\tiny, xshift=7, yshift=-7]below:$\psi_1(x_1,y_1)$}] (f1) at ($(v)!0.5!(h)$) {};
    \node[Factor, label={[font=\tiny, xshift=7, yshift=7]above:$\psi_2(x_2,y_2)$}] (f2) at ($(v1)!0.5!(h1)$) {};
    \node[Factor, label={[font=\tiny]right:$\psi_3(x_1, x_2)$}] at ($(v)!0.5!(v1)$) (f3) {};
    \factoredge {v, h} {f1} {};
    \factoredge {v1, h1} {f2} {};
    \factoredge {v1, v} {f3} {};

\end{scope}

%%% Local Variables: 
%%% mode: latex
%%% TeX-master: "../../main"
%%% End: 

        \end{tikzpicture}
        %\rule{\textwidth}{0.3pt}
        \caption{Factor graph representation for the CRF}
        \label{fig:gm-crf-example-factor}
    \end{subfigure}
    \caption[Conditional random field example]{Conditional random field example
        (\subref{fig:gm-crf-example}) with input (gray) and output variables (beige), following the
        distribution $p(y_1, y_2|x_1, x_2) = \frac{1}{Z}\phi_1(x_1, y_1)\phi_2(x_2,
        y_2)\phi(y_1,y_2)$. Its factor graph (\cref{subsec:factor-graphs}) representation is
        shown in (\subref{fig:gm-crf-example-factor}).}
    \label{fig:gm-crf}
\end{figure}


%%% Local Variables: 
%%% mode: latex
%%% TeX-master: "../../main"
%%% End: 

\subsection{Factor Graphs}
\label{subsec:factor-graphs}
Finally, to conclude the digression on graphical models, this section introduces factor graphs.

\begin{mydef}[{\citealp{kschischang_01_factor,frey_98_factor}}]
    \label{def:factor-graph}
    A \emph{factor graph} is an undirected bipartite graph $G=(\{\mathcal{U},\mathcal{V}\},
    \mathcal{E})$, \ie the set of vertices is formed by two disjoint sets $\mathcal{U}$ and
    $\mathcal{V}$ with $\mathcal{U} \cap \mathcal{V} = \emptyset$, such that there is no edge
    between two vertices from the same set $\mathcal{U}$ or $\mathcal{V}$:
\begin{align}
    \label{eq:bipartite}
    \nexists (U_1 \undirected U_2) \in \mathcal{E} \; \forall \; U_1, U_2 \in
    \mathcal{U} \wedge \nexists (V_1 \undirected V_2) \in \mathcal{E} \; \forall \; V_1, V_2 \in
    \mathcal{V}
\end{align}
It represents a probability distribution over random variables $\mathcal{X} = \{X_V : V \in \mathcal{V}\}$
\begin{align}
    \label{eq:fg-distribution}
    P(\mathcal{X}) = \frac{1}{Z}\prod_{U \in \mathcal{U}}\psi_U(\mathcal{X}_U),
\end{align}
with the vertices $\mathcal{U}$ and $\mathcal{V}$ indexing the non-negative \emph{potentials} or
\emph{factors} $\psi_\mathcal{U} = \{\psi_U : U \in \mathcal{U}\}$ and the random variables
respectively. Here, $\psi_U$ is a function of the set of random variables
$\mathcal{X}_U$. Furthermore, the set of edges $\mathcal{E}$ is defined such that the vertex $U$
representing a potential $\psi_U$ is connected to the vertices $\mathcal{V}_{\mathcal{X_U}}$ of all
of the potential's arguments $\mathcal{X}_U$:
\begin{align}
    \label{eq:fg-edges}
    \mathcal{E}_U &= \left\{(U \undirected V ) : V \in {V}_{\mathcal{X_U}}\right\}\\
    \mathcal{E} &= \left\{\mathcal{E}_U : U \in \mathcal{U}\right\}
\end{align}
\end{mydef}

With \cref{def:factor-graph}, it becomes clear, that the structure of a factor graph reflects the
factorization of a distribution into potentials. In general, the following conventions are agreed
upon for the illustrations of factor graphs:
\begin{enumerate}
      \item Random variable nodes $\mathcal{V}$ are typically denoted by circles.
      \item Factor nodes $\mathcal{U}$ are typically denoted by filled squares.
\end{enumerate}
In order to illustrate these conventions, \cref{fig:fg-simple} shows an example of a simple factor
graph with the probability distribution
\begin{align}
    \label{eq:fg-simple}
    P(x_1,x_2,x_3,x_4,x_5) = \frac{1}{Z}\psi_1(x_1,x_2,x_3)\psi_2(x_1,x_4)\psi_3(x_5).
\end{align}
Moreover, factor graphs implement dependencies and interactions of random variables in a very
intuitive manner in terms of factors. In addition, a range of inference algorithms is applicable to
factor graphs, \eg (loopy) \emph{belief propagation}~(\citealp{pearl_82_reverend};
\citealp[Chapter~5.1.3,Chapter~28.7]{barber_12_bayesian}), \emph{belief revision}
(\citealp{darwiche_96_logic}, \citealp[Chapter~5.2.1]{barber_12_bayesian}), and both
Bayesian networks and Markov networks can be transformed into extended factor graphs without loss of
expressiveness (\citealp[Section~2]{frey_03_extending}). In the case of Markov networks, factors
represent cliques in the original graph. However, multiple factors can be merged into a single
factor and then the direct mapping from cliques to factors does not hold anymore. This makes the
factor graph a powerful tool both for modeling and inference.

Furthermore, the probability distribution of a factor graph can be interpreted as a Gibbs
distribution
\begin{align}
    \label{fg-gibbs}
    P(\mathcal{X}) = \frac{1}{Z}e^{-E(\mathcal{X})}
\end{align}
with
\begin{align}
    E(\mathcal{X}) = -\log\left(\prod_{U \in \mathcal{U}}\psi_U(\mathcal{X}_U)\right) = -\sum_{U \in \mathcal{U}}\log\left(\psi_U(\mathcal{X}_U)\right).
\end{align}
Then, the MAP solution
\begin{align}
    \label{fg-map}
    \argmax_{\mathcal{X}}P(\mathcal{X}) = \argmin_{\mathcal{X}}E(\mathcal{X})
\end{align}
can be calculated using \emph{linear programming}~(LP, \cref{cha:app-ilp}) or \emph{integer linear
    programming}~(ILP, \cref{cha:app-ilp}) for discrete random
variables. In the following, we make use of this transformation to exploit the advantage that,
instead of using zero probability (infinite energy) to forbid certain configurations, we can
directly exclude these configurations from the feasible regions by the means of linear constraints
(also hard constraints).
% can be calculated using integer linear programming (ILP,
% \citet{vanderbei_07_linear}{Chapter~22}). Throughout this thesis, all presented
% tracking approaches  





\begin{figure}
    \centering
    \begin{tikzpicture}[thick, on grid, every node/.style={font=\small, scale=1.5}]
        {% \scalefont{0.5}
    \node[RVnode] (rv1) {$x_1$};
    \node[RVnode, above right=of rv1.north east, yshift=10] (rv2) {$x_2$};
    \node[RVnode, above left=of rv1.north west, yshift=10] (rv3) {$x_3$};
    \node[RVnode, below=of rv1.south] (rv4) {$x_4$};
    \node[RVnode, right=of rv2.east, xshift=20] (rv5) {$x_5$};
    \node[Factor, label={[font=\tiny]left:$\psi_2(x_1,x_4)$}] (f1) at ($(rv1)!0.5!(rv4)$) {};
    \node[Factor, label={[font=\tiny]below right:$\psi_1(x_1,x_2,x_3)$}] (f2) at ($(rv2)!0.5!(rv3)!0.3!(rv1)$) {};
    \node[Factor, below=of rv5.south, label={[font=\tiny]right:$\psi_3(x_5)$}] (f3) {};
    \factoredge {rv1, rv4} {f1} {};
    \factoredge {rv1, rv2, rv3} {f2} {};
    \factoredge {rv5} {f3} {};
    % \path[draw, very thick] (rv1) edge (f1);
    % \path[draw, very thick] (rv1) edge (f2);
    % \path[draw, very thick] (rv2) edge (f2);
    % \path[draw, very thick] (rv3) edge (f2);
    % \path[draw, very thick] (rv4) edge (f1);
    % \path[draw, very thick] (rv5) edge (f3);
}

%%% Local Variables: 
%%% mode: latex
%%% TeX-master: "../../../main"
%%% End: 

    \end{tikzpicture}
    %\rule{\textwidth}{0.3pt}
    \caption[Example of a simple factor graph]{Example of a simple factor graph containing five
        random variables (circles with beige fill) and three factors (black squares).}
    \label{fig:fg-simple}
\end{figure}




%%% Local Variables: 
%%% mode: latex
%%% TeX-master: "../../main"
%%% End: 


We introduced three probabilistic graphical models, namely Bayesian networks, conditional random
fields and factor graphs. In the following, \cref{cha:gm-in-tracking} shows two examples for
graphical model based tracking-by-assignment methods, \emph{chain graph} tracking in
\cref{subsec:fg-chaingraph} and \emph{conservation} tracking in \cref{subsec:fg-conservation}.

%%% Local Variables: 
%%% mode: latex
%%% TeX-master: "../../main"
%%% End: 
